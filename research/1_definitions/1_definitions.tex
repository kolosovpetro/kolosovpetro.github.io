\documentclass[12pt, letterpaper]{amsart}
\usepackage[left=1in,right=1in,bottom=1in,top=1in]{geometry}
\usepackage{amsfonts}
\usepackage{amsmath, amssymb}
\usepackage[font=small,labelfont=bf]{caption}
\usepackage[pdfpagelabels,hyperindex,colorlinks=true,linkcolor=blue,urlcolor=magenta,citecolor=green]{hyperref}
\usepackage{amsthm}
\usepackage{float}
\usepackage{mathrsfs}
\usepackage{colonequals}

\newtheorem{thm}{Theorem}[section]
\newtheorem{cor}[thm]{Corollary}
\newtheorem{prop}[thm]{Proposition}
\newtheorem{lem}[thm]{Lemma}
\newtheorem{conj}[thm]{Conjecture}
\newtheorem{quest}[thm]{Question}
\newtheorem{ppty}[thm]{Property}
\newtheorem{ppties}[thm]{Properties}
\newtheorem{claim}[thm]{Claim}

\theoremstyle{definition}
\newtheorem{defn}[thm]{Definition}
\newtheorem{defns}[thm]{Definitions}
\newtheorem{con}[thm]{Construction}
\newtheorem{exmp}[thm]{Example}
\newtheorem{exmps}[thm]{Examples}
\newtheorem{notn}[thm]{Notation}
\newtheorem{notns}[thm]{Notations}
\newtheorem{addm}[thm]{Addendum}
\newtheorem{exer}[thm]{Exercise}
\newtheorem{limit}[thm]{Limitation}

\theoremstyle{remark}
\newtheorem{rem}[thm]{Remark}
\newtheorem{rems}[thm]{Remarks}
\newtheorem{warn}[thm]{Warning}
\newtheorem{sch}[thm]{Scholium}

\newenvironment{myitemize}
{ \begin{itemize}
    \setlength{\itemsep}{4pt}
    \setlength{\parskip}{4pt}
    \setlength{\parsep}{4pt}     }
{ \end{itemize}                  }

\makeatletter
\let\c@equation\c@thm
\raggedbottom
\makeatother
\numberwithin{equation}{section}
%--------Meta Data: Fill in your info------
\title{Definitions}
\date{\today}
\hypersetup{
pdftitle={Definitions},
pdfsubject={11},
pdfauthor={Petro Kolosov},
pdfkeywords={11}
}
\begin{document}
\section{Notation and conventions}
We now set the following notation, which remains fixed for the remainder of this paper:

\begin{myitemize}
\item We strongly believe to D. Knuth's words in [citation]
\begin{quote}
\textit{I  realized long ago that "boundary conditions" on indices of summation are often a handicap and a waste of time.}
\end{quote}
For example, instead of writing
\begin{equation*}
2^n=\sum_{k=0}^n \binom{n}{k}
\end{equation*}
it is much better to write
\begin{equation*}
2^n=\sum_{k} \binom{n}{k}
\end{equation*}
the sum now extends over all integers $k$, but only finitely many terms are nonzero.

\item We believe to [citation] that exponential function $0^x$ should be defined for all $x$  as
\begin{equation*}
0^x = 1.
\end{equation*}

\item Iverson's convention $[P(k)]$, where $P(k)$ is logical sentence depending on $k$
\begin{equation*}
[P(k)] =
\begin{cases}
1, &\quad P(k) \ is \ true, \\
0, &\quad otherwise
\end{cases}
\end{equation*}

\item $f^{r}(n)$ is a power function defined piecewise by means of Iverson's convention
\begin{equation*}
f^{r}(n) \colonequals n^r[n\geq 0], \quad n\in\mathbb{N}.
\end{equation*}

\item Discrete convolution transform $(f\ast f)[n]$ of function $f(n)$

\begin{equation*}
(f\ast f)[n]=\sum_{k}f[k]f[n-k].
\end{equation*}

\item $\mathbf{A}_{m,r}$ is a real coefficient defined recursively as

\begin{equation*}
\mathbf{A}_{m,r}\colonequals
\begin{cases}
(2r+1)\binom{2r}{r}, & \mathrm{if } \ r=m, \\
(2r+1)\binom{2r}{r} \sum\limits_{d=2r+1}^{m} \mathbf{A}_{m,d} \binom{d}{2r+1} \frac{(-1)^{d-1}}{d-r} B_{2d-2r}, & \mathrm{if } \ 0 \leq r < m, \\
0, & \mathrm{if } \ r<0 \ \mathrm{or } \ r>m.
\end{cases}
\end{equation*}
where $B_t$ are Bernoulli numbers. We assume that $B_1=\tfrac12$. For $m\geq 11$ the $\mathbf{A}_{m,r}$ takes the fractional values for certain $r$.

\item $\mathbf{Q}^{r}_{a,b}(n)$ is the power sum defined as

\begin{equation*}
\mathbf{Q}^{r}_{a,b}(n)\colonequals\sum\limits_{a\leq k<b} k^r(n-k)^r, \quad (n,r)\in\mathbb{Z}.
\end{equation*}
Notation $\mathbf{Q}^{r}_{b}(n)$ is an equivalent to $\mathbf{Q}^{r}_{a,b}(n)$ with set $a=0$, i.e $\mathbf{Q}^{r}_{b}(n) \equiv \mathbf{Q}^{r}_{0,b}(n)$.

\item $S_p(n)$ is a common power sum [possible cite of mathworld]
\begin{equation*}
S_p(n) = \sum_{0\leq k < n} k^p.
\end{equation*}

\item $\mathbf{L}_m(n,k)$ are polynomials of degree $2m$ in $n,k$ defined involving coefficients $\mathbf{A}_{m,r}$

\begin{equation*}
\mathbf{L}_m(n,k)\colonequals\sum\limits_{r} \mathbf{A}_{m,r} k^r(n-k)^r, \quad m\in\mathbb{N}, \quad (n,k)\in\mathbb{Z}.
\end{equation*}

\item $\mathbf{P}^{m}_{a,b}(n)$ are polynomials of degree $2m$ in $a, b, n$. Polynomials $\mathbf{P}^{m}_{a,b}(n)$ are defined as a sum of $\mathbf{L}_m(n,k)$ over $k\in [a,..,b]$,

\begin{equation*}
\mathbf{P}^{m}_{a,b}(n)\colonequals \sum\limits_{a\leq k<b} \mathbf{L}_{m}(n,k), \quad n\in\mathbb{Z}.
\end{equation*}
Notation $\mathbf{P}^{m}_{b}(n)$ is an equivalent to $\mathbf{P}^{m}_{a,b}(n)$ with set $a=0$, i.e $\mathbf{P}^{m}_{b}(n) \equiv \mathbf{P}^{m}_{0,b}(n)$.

\item $\mathbf{X}^{m}_{t}(a,b)$ are polynomials of degree $2m-t$ in $a,b$ defined as

\begin{equation*}
\mathbf{X}^{m}_{t}(a,b)\colonequals (-1)^m \sum\limits_{j\geq t} \mathbf{A}_{m,j} (-1)^j \binom{j}{t} \sum\limits_{a\leq k<b}k^{2j-t}, \quad 0\leq t\leq m.
\end{equation*}
Notation $\mathbf{X}^{m}_{t}(b)$ is an equivalent to $\mathbf{X}^{m}_{t}(a,b)$ with set $a=0$, i.e $\mathbf{X}^{m}_{t}(b) \equiv \mathbf{X}^{m}_{t}(0,b)$.

\item $\mathbf{H}_{m,t}(k)$ are real coefficients defined in terms of Bernoulli numbers $B_t$, Binomial coefficients $\binom{t}{k}$ and $\mathbf{A}_{m,r}$

\begin{equation*}
\mathbf{H}_{m,t}(k)\colonequals \sum\limits_{j\geq t}\binom{j}{t}\mathbf{A}_{m,j} \frac{(-1)^j}{2j-t+1}\binom{2j-t+1}{k}B_{2j-t+1-k}, \quad 0\leq t\leq m,
\end{equation*}
where $B_1=\tfrac12$.

\end{myitemize}
\end{document}
