\documentclass[12pt,letterpaper,oneside,reqno]{amsart}
\usepackage{amsfonts}
\usepackage{amsmath}
\usepackage{amssymb}
\usepackage{amsthm}
\usepackage{float}
\usepackage{mathrsfs}
\usepackage{colonequals}
\usepackage[font=small,labelfont=bf]{caption}
\usepackage[left=1in,right=1in,bottom=1in,top=1in]{geometry}
\usepackage[pdfpagelabels,hyperindex,colorlinks=true,linkcolor=blue,urlcolor=magenta,citecolor=green]{hyperref}
\emergencystretch=1em

\newcommand \anglePower [2]{\langle #1 \rangle \sp{#2}}
\newcommand \bernoulli [2][B] {{#1}\sb{#2}}
\newcommand \multifoldSum [2][x]{{#1}\sb{1} + {#1}\sb{2} + \cdots + {#1}\sb{#2}}
\newcommand \curvePower [2]{\{#1\}\sp{#2}}
\newcommand \iversonBracket [1][s]{[#1 \; \mathrm{is} \; \mathrm{even}]}
\newcommand \floor [1]{\lfloor #1 \rfloor}
\newcommand \coeffA [3][A] {{\mathbf{#1}} \sb{#2,#3}}
\newcommand \coeffH [4][H] {{\mathbf{#1}} \sb{#2,#3} (#4)}
\newcommand \polynomialX [4][X] {{\mathbf{#1}}\sb{#2,#3} (#4)}
\newcommand \polynomialP [4][P]{{\mathbf{#1}}\sp{#2} \sb{#3}(#4)}
\newcommand \polynomialL [4][L]{{\mathbf{#1}}\sb{#2}(#3,#4)}

\newtheorem{thm}{Theorem}[section]
\newtheorem{cor}[thm]{Corollary}
\newtheorem{prop}[thm]{Proposition}
\newtheorem{lem}[thm]{Lemma}
\newtheorem{ppty}[thm]{Property}

\numberwithin{equation}{section}
%--------Meta Data: Fill in your info------
\title[On the link between Binomial Theorem and Disc. Conv. of Power Function]
{On the link between Binomial Theorem and Discrete Convolution of Power Function}
\author[Petro Kolosov]{Petro Kolosov}
\email{kolosovp94@gmail.com}
\keywords{Binomial theorem, Discrete convolution, Power function, Polynomials}
\urladdr{https://kolosovpetro.github.io}
\subjclass[2010]{44A35 (primary), 11C08 (secondary)}
\date{\today}
\hypersetup{
    pdftitle={On the link between Binomial Theorem and Discrete Convolution of Power Function},
    pdfsubject={Discrete Mathematics, Number Theory, Combinatorics},
    pdfauthor={Petro Kolosov},
    pdfkeywords={Binomial theorem, Discrete convolution, Power function, Polynomials, Convolution,
    Multinomial theorem, Binomial coefficient, Bernoulli number, Pascal's triangle, Faulhaber's formula,
    Power sum, Worpitzky identity, Macaulay's method, Macaulay brackets, Iverson's bracket, Binomial expansion}
}
\begin{document}
    \begin{abstract}
        In this manuscript we introduce and discuss the $2m+1$-degree integer valued polynomials $\polynomialP{m}{b}{n}$.
        These polynomials are in strong relation with discrete convolution of power function.
        It is also shown that odd binomial expansion is partial case of $\polynomialP{m}{b}{n}$.
        Basis on above, we show the relation between Binomial theorem and discrete
        convolution of power function.
    \end{abstract}
    \maketitle
    \tableofcontents


    \section{Introduction}
    The polynomials $\polynomialP{m}{b}{n}$ are $2m+1$-degree integer valued polynomials, which connect discrete
    convolution \cite{DiscConv} of power with Binomial theorem \cite{AbraSteg72}.
    Basically, Binomial expansion is a partial case of $\polynomialP{m}{b}{n}$, while polynomials $\polynomialP{m}{b}{n}$
    is in relation with discrete convolution of power function.

    \subsection{Definitions, Notations and Conventions}
    We now set the following notation, which remains fixed for the remainder of this paper:
    \begin{itemize}
        \item $\coeffA{m}{r}, \; m\in\mathbb{N}$ is a real coefficient defined recursively
        \begin{equation}
            \label{def_coeff_a}
            \coeffA{m}{r} \colonequals
            \begin{cases}
                (2r+1) \binom{2r}{r}, & \text{if } r=m; \\
                (2r+1) \binom{2r}{r} \sum_{d=2r+1}^{m} \coeffA{m}{d} \binom{d}{2r+1} \frac{(-1)^{d-1}}{d-r}
                \bernoulli{2d-2r}, & \text{if } 0 \leq r<m; \\
                0, & \text{if } r<0 \text{ or } r>m;
            \end{cases}
        \end{equation}
        where $\bernoulli{t}$ are Bernoulli numbers \cite{WeissteinBernoulli}.
        It is assumed that $\bernoulli{1}=\frac{1}{2}$.
        \item $\polynomialL{m}{n}{k}, \; m\in\mathbb{N}, \; n,k\in\mathbb{R}$ is polynomial of degree $2m$ in $n,k$
        \begin{equation*}
            \label{def_polynomial_l}
            \polynomialL{m}{n}{k} \colonequals \sum_{r=0}^{m} \coeffA{m}{r} k^r(n-k)^r
        \end{equation*}
        \item $\polynomialP{m}{b}{n}, \; m,b\in\mathbb{N}, \; n\in\mathbb{R}$ is polynomial of degree $2m+1$ in $b, n$
        \begin{equation*}
            \label{def_polynomial_p}
            \polynomialP{m}{b}{n} \colonequals \sum_{k=0}^{b-1} \polynomialL{m}{n}{k}
        \end{equation*}
        \item $\coeffH{m}{t}{b}, \; m,t,b\in\mathbb{N}$ is a natural coefficient defined as
        \begin{equation*}
            \label{def_coeff_h}
            \coeffH{m}{t}{b}
            \colonequals
            \sum_{j=t}^{m} \binom{j}{t} \coeffA{m}{j} \frac{(-1)^j}{2j-t+1} \binom{2j-t+1}{b} \bernoulli{2j-t+1-b}
        \end{equation*}
        \item $\polynomialX{m}{t}{j}, \; m,t\in\mathbb{N}, \; j\in\mathbb{R}$ is polynomial of degree $2m+1-t$ in $j$
        \begin{equation*}
            \label{def_coeff_x}
            \polynomialX{m}{t}{j} \colonequals (-1)^m \sum_{k=1}^{2m+1-t} \coeffH{m}{t}{k} \cdot j^k
        \end{equation*}
        \item $[P(k)]$ is the Iverson's convention \cite{APL}, where $P(k)$ is logical sentence on $k$
        \begin{equation}
            \label{def_iverson}
            [P(k)] =
            \begin{cases}
                1, & \quad P(k) \; \mathrm{is} \; \mathrm{true} \\
                0, & \quad \mathrm{otherwise}
            \end{cases}
        \end{equation}
        \item $\anglePower{x-a}{n}$ is powered Macaulay bracket, \cite{Mac19}
        \begin{equation}
            \label{def_macaulay_bracket}
            \anglePower{x-a}{n} \colonequals
            \begin{cases}
                (x-a)^n, &\quad x\geq a \\
                0, &\quad \mathrm{otherwise}
            \end{cases} \quad a\in\mathbb{Z}
        \end{equation}
        \item $\curvePower{x-a}{n}$ is powered Macaulay bracket
        \begin{equation}
            \label{def_macaulay_bracket_strict}
            \curvePower{x-a}{n} \colonequals
            \begin{cases}
                (x-a)^n, &\quad x>a \\
                0, &\quad \mathrm{otherwise}
            \end{cases}  \quad a\in\mathbb{Z}
        \end{equation}
        During the manuscript, the variable $a$ is reserved to be only the condition of Macaulay functions.
        If the power function $\anglePower{x-a}{n}$ or $\curvePower{x-a}{n}$ is written without parameter $a$
        e.g $\anglePower{x}{n}$ or $\curvePower{x}{n}$ means that it is assumed $a=0$.
    \end{itemize}


    \section{Polynomials \texorpdfstring{$\polynomialP{m}{b}{n}$}{P[m,b,n]} and their properties}
    We'd like to begin our discussion from short overview of polynomial $\polynomialL{m}{n}{k}$.
    Polynomial $\polynomialL{m}{n}{k}, \; m\in\mathbb{N}, \; n,k \in \mathbb{R}$ is polynomial of degree $2m$ in $n,k$.
    In extended form, the polynomial $\polynomialL{m}{n}{k}$ is as follows
    \begin{equation*}
        \polynomialL{m}{n}{k} = \coeffA{m}{m} k^m(n-k)^m + \coeffA{m}{m-1} k^{m-1}(n-k)^{m-1}+\cdots+\coeffA{m}{0},
    \end{equation*}
    where $\coeffA{m}{r}$ are real coefficients by definition~\ref{def_coeff_a}.
    Note that $\polynomialL{m}{n}{k}$ implicitly involves discrete convolution of power function.
    We'll back to it soon.
    The coefficient $\coeffA{m}{r}$ is a real number.
    Coefficients $\coeffA{m}{r}$ are nonzero only for $r$ within the interval $r \in \{m\} \cup \left[0,\frac{m-1}{2}\right]$.
    For example, let's show an array of $\coeffA{m}{r}$ for $m=0,1,2,.. 7$.
    In order to proceed, let's type \textit{Column[Table[CoeffA[m, r], \{m, 0, 7\}, \{r, 0, m\}], Left]}
    into Mathematica console using \cite{mmca_package}, we get
    \begin{table}[H]
        \begin{tabular}{c|cccccccc}
            $m/r$ &0 &1 &2 &3 &4 &5 &6 &7 \\ [3px]
            \hline
            0 &1 & & & & & & & \\
            1 &1 &6 & & & & & & \\
            2 &1 &0 &30 & & & & & \\
            3 &1 &-14 &0 &140 & & & & \\
            4 &1 &-120 &0 &0 &630 & & & \\
            5 &1 &-1386 &660 &0 &0 &2772 & & \\
            6 &1 &-21840 &18018 &0 &0 &0 &12012 & \\
            7 &1 &-450054 &491400 &-60060 &0 &0 &0 &51480
        \end{tabular}
        \caption{Coefficients $\coeffA{m}{r}$.
        For $m \geq 11$ the real values of $\coeffA{m}{r}$ are taking place.}
    \end{table}
    Thus, the polynomial $\polynomialL{m}{n}{k}$ could be written as well as
    \begin{equation*}
        \polynomialL{m}{n}{k} = \coeffA{m}{m} k^m (n-k)^m + \sum_{r=0}^{\frac{m-1}{2}} \coeffA{m}{r} k^r (n-k)^r
    \end{equation*}
    For example, a few of polynomials $\polynomialL{m}{n}{k}$ are
    \begin{equation*}
        \begin{split}
            \polynomialL{0}{n}{k}
            &= 1, \\
            \polynomialL{1}{n}{k}
            &= 6 k (n-k) + 1
            = -6 k^2 + 6 k n + 1, \\
            \polynomialL{2}{n}{k}
            &=30 k^2 (n-k)^2+1
            =30 k^4-60 k^3 n+30 k^2 n^2+1, \\
            \polynomialL{3}{n}{k}
            &= 140 k^3 (n-k)^3-14 k (n-k)+1 \\
            &=-140 k^6+420 k^5 n-420 k^4 n^2+140 k^3 n^3+14 k^2-14 k n+1
        \end{split}
    \end{equation*}
    We'd like to notice that the polynomials $\polynomialL{m}{n}{k}$ are symmetrical as follows
    \begin{ppty}
        \label{ppty_symmetry_of_polynomial_l}
        For every $n,k\in\mathbb{Z}$
        \begin{equation*}
            \polynomialL{m}{n}{k} = \polynomialL{m}{n-k}{k}
        \end{equation*}
    \end{ppty}
    Following table displays the symmetry of $\polynomialL{m}{n}{k}$
    \begin{table}[H]
        \begin{tabular}{c|cccccccc}
            $n/k$ &0 &1 &2 &3 &4 &5 &6 &7 \\ [3px]
            \hline
            0 &1 & & & & & & & \\
            1 &1 &1 & & & & & & \\
            2 &1 &7 &1 & & & & & \\
            3 &1 &13 &13 &1 & & & & \\
            4 &1 &19 &25 &19 &1 & & & \\
            5 &1 &25 &37 &37 &25 &1 & & \\
            6 &1 &31 &49 &55 &49 &31 &1 & \\
            7 &1 &37 &61 &73 &73 &61 &37 &1
        \end{tabular}
        \caption{Triangle generated by $\polynomialL{1}{n}{k}, \; 0 \leq k \leq n$.}
        \label{fig_1}
    \end{table}
    Therefore, we have briefly discussed the polynomials $\polynomialL{m}{n}{k}$, which is required step to be
    familiarized with the polynomial $\polynomialP{m}{b}{n}$ in detailed way.
    For now, let's discuss the polynomial $\polynomialP{m}{b}{n}$.
    Polynomial $\polynomialP{m}{b}{n}$ is defined as summation of $\polynomialL{m}{n}{k}$ over $k$
    such that $0\leq k \leq b-1$.
    In extended form, the polynomial $\polynomialP{m}{b}{n}$ is
    \begin{equation*}
        \begin{split}
            \polynomialP{m}{b}{n}
            &=\sum_{k=0}^{b-1} \polynomialL{m}{n}{k}
            =\sum_{k=0}^{b-1} \sum_{r=0}^{m} \coeffA{m}{r} k^r(n-k)^r
            =\sum_{r=0}^{m} \coeffA{m}{r} \sum_{k=0}^{b-1} k^r(n-k)^r \\
            &=\sum_{r=0}^{m} \coeffA{m}{r} \sum_{k=0}^{b-1} k^r \sum_{j=0}^{r} (-1)^j \binom{r}{j} n^{r-j} k^j \\
            &=\sum_{r=0}^{m} \sum_{j=0}^{r} (-1)^j n^{r-j} \binom{r}{j} \coeffA{m}{r} \sum_{k=0}^{b-1} k^{r+j} \\
            &=\sum_{r=0}^{m} \sum_{j=0}^{r} n^{r-j} \binom{r}{j} \coeffA{m}{r} \frac{(-1)^j}{r+j+1} \sum_{s=0}^{r+j}
            \binom{r+j+1}{s} \bernoulli{s} (b-1)^{r+j-s+1}
        \end{split}
    \end{equation*}
    However, by Property~\eqref{ppty_symmetry_of_polynomial_l}, the $\polynomialP{m}{b}{n}$ may be written in the form
    \begin{equation*}
        \begin{split}
            \polynomialP{m}{b}{n}
            &=\sum_{k=1}^{b} \polynomialL{m}{n}{k}
            =\sum_{k=1}^{b} \sum_{r=0}^{m} \coeffA{m}{r} k^r(n-k)^r \\
            &=\sum_{k=1}^{b} \sum_{r=0}^{m} \coeffA{m}{r} k^r \sum_{t=0}^{r} (-1)^{r-t} n^t \binom{r}{t} k^{r-t} \\
            &=\sum_{t=0}^{m} n^t \underbrace{\sum_{k=1}^{b} \sum_{r=t}^{m} (-1)^{r-t} \binom{r}{t}
                \coeffA{m}{r} k^{2r-t}}_{(-1)^{m-t} \polynomialX{m}{t}{b}}
        \end{split}
    \end{equation*}
    In above formula brace underlines $2m+1-t$ degree polynomial
    $\polynomialX{m}{t}{j}, \; m,t\in\mathbb{N}, \; j\in\mathbb{R}$ in $j$, see definition ~\eqref{def_coeff_x}.
    From this formula it may be not immediately clear why $\polynomialX{m}{t}{j}$ represent polynomials in $j$.
    However, this can be seen if we change the summation order and use Faulhaber's formula $\sum _{k=1}^{n} k^{p} =
    \frac{1}{p+1}\sum _{j=0}^{p} \binom{p+1}{j} \bernoulli{j} n^{p+1-j}$ to obtain
    \begin{equation*}
        \polynomialX{m}{t}{j} = (-1)^m \sum_{r=t}^{m} \binom{r}{t} \coeffA{m}{r} \frac{(-1)^r}{2r-t+1}
        \sum_{\ell=0}^{2r-t} \binom{2r-t+1}{\ell} \bernoulli{\ell} j^{2r-t+1-\ell}
    \end{equation*}
    Introducing $k=2r-t+1-\ell$, we further get the formula
    \begin{equation*}
        \polynomialX{m}{t}{j} = (-1)^m \sum_{k=1}^{2m-t+1} j^k
        \underbrace{\sum_{r=t}^m \binom{r}{t} \coeffA{m}{r} \frac{(-1)^r}{2r-t+1} \binom{2r-t+1}{k}
            \bernoulli{2r-t+1-k}}_{\coeffH{m}{t}{k}}
    \end{equation*}
    To produce examples of $\polynomialX{m}{t}{j}$ for $m=3$ let's type
    \textit{Column[Table[X[3, n, j], \{n, 0, 3\}], Left]} into Mathematica console using \cite{mmca_package},
    we get further
    \begin{equation*}
        \begin{split}
            \polynomialX{3}{0}{j}
            &=7 j^2 - 28 j^3 + 70 j^5 - 70 j^6 + 20 j^7, \\
            \polynomialX{3}{1}{j}
            &=7 j - 42 j^2 + 175 j^4 - 210 j^5 + 70 j^6, \\
            \polynomialX{3}{2}{j}
            &=-14 j + 140 j^3 - 210 j^4 + 84 j^5, \\
            \polynomialX{3}{3}{j}
            &=35 j^2 - 70 j^3 + 35 j^4
        \end{split}
    \end{equation*}
    It gives us opportunity to review the $\polynomialP{m}{b}{n}$ from different prospective, for instance
    \begin{equation}
        \label{p_all_forms}
        \polynomialP{m}{b}{n}
        =\sum_{r=0}^{m} (-1)^{m-r} \polynomialX{m}{r}{b} \cdot n^r
        =\sum_{r=0}^{m} \sum_{\ell=1}^{2m-r+1} (-1)^{2m-r} \coeffH{m}{r}{\ell} \cdot b^\ell \cdot n^r
    \end{equation}
    The last line of \eqref{p_all_forms} clearly states why $\polynomialP{m}{b}{n}$ are polynomials in $b,n$.
    Let's show a few examples of polynomials $\polynomialP{m}{b}{n}$
    \begin{equation*}
        \begin{split}
            \polynomialP{0}{b}{n}
            &=b, \\
            \polynomialP{1}{b}{n}
            &=3 b^2 - 2 b^3 - 3 b n + 3 b^2 n, \\
            \polynomialP{2}{b}{n}
            &=10 b^3 - 15 b^4 + 6 b^5 \\
            &- 15 b^2 n + 30 b^3 n - 15 b^4 n \\
            &+ 5 b n^2 - 15 b^2 n^2 + 10 b^3 n^2, \\
            \polynomialP{3}{b}{n}
            &=-7 b^2 + 28 b^3 - 70 b^5 + 70 b^6 - 20 b^7 \\
            &+ 7 b n - 42 b^2 n + 175 b^4 n - 210 b^5 n + 70 b^6 n \\
            &+ 14 b n^2 - 140 b^3 n^2 + 210 b^4 n^2 - 84 b^5 n^2 \\
            &+ 35 b^2 n^3 - 70 b^3 n^3 + 35 b^4 n^3
        \end{split}
    \end{equation*}
    The following property also holds in terms of $\polynomialP{m}{b}{n}$
    \begin{ppty}
        \label{prop_p_identity}
        For every $m,b,n \in \mathbb{N}$
        \begin{equation*}
            \polynomialP{m}{b+1}{n} = \polynomialP{m}{b}{n} + \polynomialL{m}{n}{b}
        \end{equation*}
    \end{ppty}
    We consider the polynomials $\polynomialP{m}{b}{n}$ since that basis on them we reveal the main aim of the work and establish a connection between the Binomial theorem and the discrete convolution of a power
    functions $\anglePower{x}{n}$, $\curvePower{x}{n}$.
    In the next section we show that odd binomial expansion $(x+y)^{2m+1}, \; 2m+1, \; m\geq 0$ is a partial case of $\polynomialP{m}{b}{n}$.


    \section{Odd Binomial expansion as partial case of polynomial
        \texorpdfstring{$\polynomialP{m}{b}{n}$}{P[m,b,n]}}
    An odd power function $n^{2m+1}, \; n,m\in \mathbb{N}$ could be expressed in terms of partial case of
    polynomial $\polynomialP{m}{b}{n}$ as follows
    \begin{lem}
        \label{lemma_polynomial_p_and_odd_power}
        For each $n,m\in\mathbb{N}$
        \begin{equation*}
            \polynomialP{m}{n}{n} \equiv n^{2m+1}
        \end{equation*}
    \end{lem}
    By Lemma~\ref{lemma_polynomial_p_and_odd_power} and ~\eqref{p_all_forms} the following identities straightforward
    \begin{equation*}
        n^{2m+1}
        =\sum_{r=0}^{m} \sum_{\ell=1}^{2m-r+1} (-1)^{2m-r} \coeffH{m}{r}{\ell} \cdot n^{\ell+r}
        =\sum_{r=0}^{m} (-1)^{m-r} \polynomialX{m}{r}{n} \cdot n^r
    \end{equation*}
    By Property~\ref{ppty_symmetry_of_polynomial_l} an odd power $n^{2m+1} + 1, \; m,n\in\mathbb{N}$ can also be
    expressed as
    \begin{equation*}
        n^{2m+1} + 1 = \iversonBracket[n] \polynomialL{m}{n}{n/2} + 2 \sum_{k=0}^{\frac{n-1}{2}} \polynomialL{m}{n}{k},
    \end{equation*}
    where $\iversonBracket[n]$ is Iverson's bracket~\eqref{def_iverson}.
    Moreover, the partial case of Lemma~\ref{lemma_polynomial_p_and_odd_power} for $n = x+y$ gives binomial expansion
    \begin{cor}
        \label{bin_exp_by_p}
        (Partial case of Lemma~\ref{lemma_polynomial_p_and_odd_power} for binomials.)
        For every $x+y,m\in\mathbb{N}$
        \begin{equation*}
            \polynomialP{m}{x+y}{x+y} \equiv \sum_{r=0}^{m} \binom{2m+1}{r} x^{2m-r} y^r
        \end{equation*}
    \end{cor}
    For instance, for $m=2$ we have
    \begin{equation*}
        \polynomialP{2}{x+y}{x+y} = (x + y) (x^4 + 4 x^3 y + 6 x^2 y^2 + 4 x y^3 + y^4),
    \end{equation*}
    which is binomial expansion of fifth power.
    In addition, the following power identities in terms of $\coeffH{m}{r}{x}, \; \polynomialX{m}{r}{x}$ are taking place
    \begin{equation*}
        \begin{split}
            (x+y)^{2m+1}
            &=\sum_{r=0}^{m} \sum_{\ell=1}^{2m-r+1} (-1)^{2m-r} \coeffH{m}{r}{\ell} \cdot (x+y)^{\ell+r} \\
            &=\sum_{r=0}^{m} (-1)^{m-r} \polynomialX{m}{r}{x+y} \cdot (x+y)^r
        \end{split}
    \end{equation*}
    It clearly follows that Multinomial expansion of odd-powered $t$-fold sum $(\multifoldSum{t})^{2m+1}$ can be
    reached by $\polynomialP{m}{b}{\multifoldSum{t}}$ as well
    \begin{cor}
        For all $\multifoldSum{t},m \in \mathbb{N}$
        \begin{equation*}
            \polynomialP{m}{\multifoldSum{t}}{\multifoldSum{t}}
            \equiv
            \sum_{\multifoldSum[k]{t}=2m+1} \binom{2m+1}{k_1, k_2,\ldots, k_t} \prod_{s=1}^{t} x_t^{k_s}
        \end{equation*}
    \end{cor}
    Moreover, the following multinomial identities hold
    \begin{equation*}
        \begin{split}
            (\multifoldSum{t})^{2m+1}
            &=\sum_{r=0}^{m} \sum_{\ell=1}^{2m-r+1} (-1)^{2m-r} \coeffH{m}{r}{\ell} \cdot (\multifoldSum{t})^{\ell+r} \\
            &=\sum_{r=0}^{m} (-1)^{m-r} \polynomialX{m}{r}{\multifoldSum{t}} \cdot (\multifoldSum{t})^r
        \end{split}
    \end{equation*}


    \section{Binomial expansion in terms of \texorpdfstring{$\polynomialP{m}{b}{n}$}{P[m,b,n]}
        and Iverson's convention}
    In the previous section we have established a connection between the polynomial $\polynomialP{m}{b}{n}$
    and an odd power function.
    To go over to the general case of a power function, we consider the relationship between an even and an odd powers,
    namely
    \begin{equation*}
        x^{\mathrm{even}} = x \cdot x^{\mathrm{odd}}
    \end{equation*}
    Thus, a generalized power function $x^s, \; s \in \mathbb{N}$ can be expressed in terms of an odd power function
    and the Iverson's bracket as follows
    \begin{ppty}
        \label{ppty_odd_power_and_iverson}
        For every $x\in\mathbb{R}, \; s\in\mathbb{Z}$
        \begin{equation*}
            x^{s} = x^{\iversonBracket} \cdot x^{2 \floor{ \tfrac{s-1}{2} } + 1}
        \end{equation*}
    \end{ppty}
    Hereof, it is easy to obtain a power $x^s$ by means of partial case of $\polynomialP{m}{b}{n}$
    for all $s \geq 1 \in \mathbb{N}$
    \begin{cor}
        \label{cor_general_power_by_polynomial_p_and_iverson}
        By Lemma~\ref{lemma_polynomial_p_and_odd_power} and Property~\ref{ppty_odd_power_and_iverson},
        for every $x,s\geq 1 \in \mathbb{N}$
        \begin{equation*}
            x^s = x^{\iversonBracket} \polynomialP{\tfrac{s-1}{2}}{x}{x}
        \end{equation*}
    \end{cor}
    Therefore, for every $x,s\geq 1\in\mathbb{N}$ as well true
    \begin{equation*}
        \begin{split}
            x^s
            &=\sum_{k=0}^{h} \sum_{\ell=1}^{2h-k+1} (-1)^{2h-k} \coeffH{h}{k}{\ell} \cdot x^{\iversonBracket+\ell+k} \\
            &=\sum_{k=0}^{h} (-1)^{h-k} \polynomialX{h}{k}{x} \cdot x^{\iversonBracket+k},
        \end{split}
    \end{equation*}
    where $h=\frac{s-1}{2}$ and $\iversonBracket$ is Iverson's bracket.
    The binomial expansion of $(x+y)^s$ for every $s \geq 1 \in \mathbb{N}$ is
    \begin{cor}
        \label{all_power_binomial}
        By Corollary~\ref{bin_exp_by_p} and Property~\ref{ppty_odd_power_and_iverson},
        for each $s \geq 1, x, y \in \mathbb{N}$
        \begin{equation*}
            (x+y)^{\iversonBracket} \polynomialP{h}{x+y}{x+y} \equiv \sum_{k\geq0}^{} \binom{s}{k} x^{s-k} y^k,
        \end{equation*}
        where $h=\tfrac{s-1}{2}$.
    \end{cor}
    By Corollary~\ref{all_power_binomial}
    \begin{equation*}
        \begin{split}
            (x+y)^s
            &=\sum_{k=0}^{h} \sum_{\ell=1}^{2h-k+1} (-1)^{2h-k} \coeffH{h}{k}{\ell}
            \cdot (x+y)^{\iversonBracket+\ell+k} \\
            &=\sum_{k=0}^{h} (-1)^{h-k} \polynomialX{h}{k}{x+y} \cdot (x+y)^{\iversonBracket+k},
        \end{split}
    \end{equation*}
    where $h=\frac{s-1}{2}$.
    Now we are able to express the corollary~\ref{all_power_binomial} in terms of multinomials.
    For the $t$-fold $s$-powered sum $(\multifoldSum{t})^s$ we have following relation between Multinomial expansion
    and $\polynomialP{h}{x}{x}$
    \begin{cor}
        For all $s \geq 1, \; \multifoldSum{t},s \in \mathbb{N}$
        \begin{equation*}
            (\multifoldSum{t})^{\iversonBracket} \polynomialP{h}{\multifoldSum{t}}{\multifoldSum{t}}
            \equiv \sum_{\multifoldSum[k]{t}=s} \binom{s}{k_1, k_2,\ldots, k_t} \prod_{\ell=1}^{t} x_\ell^{k_\ell},
        \end{equation*}
        where $h=\frac{s-1}{2}$.
    \end{cor}
    Therefore, the following expression holds as well
    \begin{equation*}
        \begin{split}
            (\multifoldSum{t})^{s}
            &=\sum_{k=0}^{h} \sum_{\ell=1}^{2h-k+1} (-1)^{2h-k} \coeffH{h}{k}{\ell}
            \cdot (\multifoldSum{t})^{\iversonBracket+\ell+t} \\
            &=\sum_{k=0}^{h} (-1)^{h-k} \polynomialX{h}{k}{\multifoldSum{t}}
            \cdot (\multifoldSum{t})^{\iversonBracket+k},
        \end{split}
    \end{equation*}
    where $h=\frac{s-1}{2}$.


    \section{Relation between \texorpdfstring{$\polynomialP{m}{b}{n}$}{P(m,b,n)} and discrete convolution
    of power functions \texorpdfstring{$\anglePower{x}{n}$}{<x to pow n>},
        \texorpdfstring{$\curvePower{x}{n}$}{\{x to pow n\}}}
    Previously we have established the relations between polynomial $\polynomialP{m}{b}{n}$, Binomial and Multinomial
    theorems.
    In this section we establish and discuss a relation between $\polynomialP{m}{b}{n}$ and convolution of the power
    functions $\anglePower{x}{n}, \; \curvePower{x}{n}$.
    To show that $\polynomialP{m}{b}{n}$ implicitly involves the discrete convolution of the
    power function $\anglePower{x}{n}$ let's remind that
    \begin{equation*}
        \polynomialP{m}{b}{n} = \sum_{r=0}^{m} \coeffA{m}{r} \sum_{k=0}^{b-1} k^r (n-k)^r
    \end{equation*}
    A discrete convolution of defined over set of integers $\mathbb{Z}$ function $f$ is following
    \begin{equation*}
        (f \ast f)[n] = \sum_{k} f(k) f(n-k)
    \end{equation*}
    Now a general formula of discrete convolution $\anglePower{x}{n} \ast \anglePower{x}{n}$ could be derived
    immediately
    \begin{equation*}
        \begin{split}
            \anglePower{x-a}{n} \ast \anglePower{x-a}{n}
            &=\sum_{k} \anglePower{k-a}{n} \anglePower{x-k-a}{n} \\
            &=\sum_{k} (k-a)^n (x-k-a)^n [k\geq a][x-k\geq a] \\
            &=\sum_{k} (k-a)^n (x-k-a)^n [k\geq a][k\leq x-a] \\
            &=\sum_{k} (k-a)^n (x-k-a)^n [a \leq k \leq x-a] \\
            &=\sum_{k=a}^{x-a} (k-a)^n (x-k-a)^n,
        \end{split}
    \end{equation*}
    where $[a \leq k \leq x-a]$ is Iverson's bracket of $k$.
    Note that $a$ is parameter of $\anglePower{x-a}{r}$ by definition \eqref{def_macaulay_bracket}.
    In above equation we have applied a Knuth's recommendation \cite{Knuth92} concerning the sigma notation of sums.
    Thus, we have a general expression of discrete convolution $\anglePower{x}{n} \ast \anglePower{x}{n}$.
    For now, let's notice that
    \begin{lem}
        \label{lemma_disc_conv_identity}
        For every $x\geq 1, \; x\in\mathbb{N}$
        \begin{equation*}
            \anglePower{x}{n} \ast \anglePower{x}{n} \equiv \sum_{k=0}^{x} k^r (n-k)^r
        \end{equation*}
    \end{lem}
    Thus,
    \begin{cor}
        \label{cor_polynomial_p_and_macaulay_convolution}
        By Lemma~\ref{lemma_disc_conv_identity}, the polynomials $\polynomialP{m}{b}{n}$ is in relation with discrete
        convolution of power function $\anglePower{x}{n} \ast \anglePower{x}{n}$ as follows
        \begin{equation*}
            \polynomialP{m}{x+1}{x} = \sum_{r=0}^{m} \coeffA{m}{r} \anglePower{x}{r} \ast \anglePower{x}{r}
        \end{equation*}
    \end{cor}
    Therefore, another conclusion follows
    \begin{thm}
        \label{thm_odd_power_by_macaulays_convolution}
        By Lemma~\ref{lemma_polynomial_p_and_odd_power}, Corollary~\ref{cor_polynomial_p_and_macaulay_convolution}
        and property~\ref{prop_p_identity}, for every $x\geq 1, \; x,m\in\mathbb{N}$
        \begin{equation*}
            x^{2m+1} = -1 + \sum_{r=0}^{m} \coeffA{m}{r} \anglePower{x}{r} \ast \anglePower{x}{r}
        \end{equation*}
    \end{thm}
    As next step, let's show a relation between discrete convolution of $\curvePower{x}{n}$ and power function
    of odd exponent.
    Discrete convolution $\curvePower{x-a}{n} \ast \curvePower{x-a}{n}$ is following
    \begin{equation*}
        \begin{split}
            \curvePower{x-a}{n} \ast \curvePower{x-a}{n}
            &=\sum_{k} \curvePower{k-a}{n} \curvePower{x-k-a}{n} \\
            &=\sum_{k} (k-a)^n (x-k-a)^n [k > a] [x-k > a] \\
            &=\sum_{k} (k-a)^n (x-k-a)^n [a < k < x-a] \\
            &=\sum_{k} (k-a)^n (x-k-a)^n [a+1 \leq k \leq x-a-1] \\
            &=\sum_{k=a+1}^{x-a-1} (k-a)^n (x-k-a)^n
        \end{split}
    \end{equation*}
    Now we notice the following identity in terms of polynomial $\polynomialP{m}{b}{n}$ and
    discrete convolution $\curvePower{x}{n} \ast \curvePower{x}{n}$
    \begin{prop}
        \label{prop_polynomial_p_and_macaulay_convolution_strict}
        For every $m,b,n \in \mathbb{N}$
        \begin{equation*}
            \begin{split}
                \polynomialP{m}{x}{x}
                &=\sum_{r=0}^{m} \coeffA{m}{r} \left(0^r x^r + \sum_{k=1}^{x-1} k^r (x-k)^r \right) \\
                &=\sum_{r=0}^{m} \coeffA{m}{r} 0^r x^r + \sum_{r=0}^{m} \coeffA{m}{r}
                \curvePower{x}{r} \ast \curvePower{x}{r}
            \end{split}
        \end{equation*}
    \end{prop}
    Since that for all $r$ in $\coeffA{m}{r} 0^r x^r$ we have
    \begin{equation*}
        \coeffA{m}{r} 0^r x^r =
        \begin{cases}
            1, & \mbox{if } r=0 \\
            0, & \mbox{if } r>0
        \end{cases}
    \end{equation*}
    Above is true because $\coeffA{m}{0}=1$ for every $m\in\mathbb{N}$, it is convinced \cite{Grah94SN} that $x^0 = 1$
    for every $x$.
    Hence, the following identity between $\polynomialP{m}{b}{n}$ and
    discrete convolution $\curvePower{x}{n} \ast \curvePower{x}{n}$ holds
    \begin{thm}
        \label{thm_odd_power_by_macaulays_convolution_strict}
        By Lemma~\ref{lemma_polynomial_p_and_odd_power} and
        Proposition~\ref{prop_polynomial_p_and_macaulay_convolution_strict},
        for every $x\geq 1, \; x,m\in\mathbb{N}$
        \begin{equation*}
            x^{2m+1} = 1 + \sum_{r=0}^{m} \coeffA{m}{r} \curvePower{x}{r} \ast \curvePower{x}{r}
        \end{equation*}
    \end{thm}
    \begin{cor}
        \label{cor_sum_of_coeffs_a}
        By Theorem~\ref{thm_odd_power_by_macaulays_convolution_strict}, for all $m\in\mathbb{N}$
        \begin{equation*}
            \sum_{r=0}^{m} \coeffA{m}{r} = 2^{2m+1} - 1
        \end{equation*}
    \end{cor}
    Corollary~\ref{cor_sum_of_coeffs_a} holds since that convolution $\curvePower{x}{r} \ast \curvePower{x}{r}=1$
    for each $r$ when $x=2$.
    Furthermore, we are able to find a relation between the Binomial theorem and discrete convolution of
    power function $\anglePower{x}{n}$.


    \section{Binomial expansion in terms of discrete convolutions of power function
        \texorpdfstring{$\anglePower{x}{n}$}{<x to pow n>}, \texorpdfstring{$\curvePower{x}{n}$}{\{x to pow n\}}}
    The equivalence \eqref{all_power_binomial} states the relation between Binomial theorem and partial case
    of $\polynomialP{m}{b}{n}$.
    As it is stated previously in Corollary~\ref{cor_polynomial_p_and_macaulay_convolution} and
    Proposition~\ref{prop_polynomial_p_and_macaulay_convolution_strict}, the polynomials $\polynomialP{m}{b}{n}$
    are able to be expressed in terms of discrete convolutions $\anglePower{x}{n} \ast \anglePower{x}{n}$,
    $\curvePower{x}{n} \ast \curvePower{x}{n}$.
    In this section we generalize these results in order to show relation between Binomial, Multinomial theorems and
    discrete convolutions $\anglePower{x}{n} \ast \anglePower{x}{n}$, $\curvePower{x}{n} \ast \curvePower{x}{n}$
    \begin{cor}
        \label{cor_bin_exp_and_macaulay_conv}
        (Partial case of Theorem~\ref{thm_odd_power_by_macaulays_convolution} for Binomials.)
        For each $x+y\geq 1, \; x,y,m\in\mathbb{N}$
        \begin{equation*}
            \sum_{r=0}^{m} \coeffA{m}{r} \anglePower{x+y}{r} \ast \anglePower{x+y}{r}
            \equiv
            1 + \sum_{r=0}^{2m+1} \binom{2m+1}{r} x^{2m+1-r} y^r
        \end{equation*}
    \end{cor}
    For example, for $m=0,1,2$ the Corollary~\ref{cor_bin_exp_and_macaulay_conv} gives
    \begin{equation*}
        \begin{split}
            \sum_{r=0}^{0} \coeffA{0}{r} \anglePower{x+y}{r} \ast \anglePower{x+y}{r}
            &= 1 + x + y \\
            \sum_{r=0}^{1} \coeffA{1}{r} \anglePower{x+y}{r} \ast \anglePower{x+y}{r}
            &= 1 + x + y + (-1 + x + 3 y - 2 y) (x + y) (1 + x + y) \\
            &= x^3 + 3 x^2 y + 3 x y^2 + y^3 + 1\\
            \sum_{r=0}^{2} \coeffA{2}{r} \anglePower{x+y}{r} \ast \anglePower{x+y}{r}
            &=1 + x + y + (x + y) (1 + x + y) \left(-1 + x + 5 x^2 + y + 10 x y + 5 y^2\right. \\
            &-15 x (x + y) + 10 x^2 (x + y) - 15 y (x + y) + 20 x y (x + y) \\
            &+ 10 y^2 (x + y) +9 (x + y)^2 - 15 x (x + y)^2 \\
            &\left.-15 y (x + y)^{2} + 6 {(x + y)}^{3}\right) \\
            &=x^5 + 5 x^4 y + 10 x^3 y^2 + 10 x^2 y^3 + 5 x y^4 + y^5 + 1
        \end{split}
    \end{equation*}
    To get above examples we were using command \textit{Simplify[Sum[CoeffA[1, r] * MacaulayDiscConv[x + y, r, 0],
        \{r, 0, 1\}], Element[x, Integers], Assumptions -\textgreater x + y \textgreater 0]} in Mathematica
    console by \cite{mmca_package}.
    \begin{cor}
        \label{cor_bin_exp_and_macaulay_conv_strict}
        (Partial case of Theorem~\ref{thm_odd_power_by_macaulays_convolution_strict} for Binomials.)
        For each $x+y\geq 1, \; x,y,m\in\mathbb{N}$
        \begin{equation*}
            \sum_{r=0}^{m} \coeffA{m}{r} \curvePower{x+y}{r} \ast \curvePower{x+y}{r}
            \equiv
            -1 + \sum_{r=0}^{2m+1} \binom{2m+1}{r} x^{2m+1-r} y^r
        \end{equation*}
    \end{cor}
    For example, for $m=0,1$ the Corollary~\ref{cor_bin_exp_and_macaulay_conv_strict} gives
    \begin{equation*}
        \begin{split}
            \sum_{r=0}^{0} \coeffA{0}{r} \curvePower{x+y}{r} \ast \curvePower{x+y}{r}
            &= x + y - 1 \\
            \sum_{r=0}^{1} \coeffA{1}{r} \curvePower{x+y}{r} \ast \curvePower{x+y}{r}
            &= x + y - 1 - (x + y - 1) (x + y) (2 (x + y)-1 - 3 x - 3 y) \\
            &= x^3 + 3 x^2 y + 3 x y^2 + y^3 - 1
        \end{split}
    \end{equation*}
    In case of variations of $a$ in $\anglePower{x-a}{n}$ and $\curvePower{x-a}{n}$
    for each $x \geq 2a, \; a = \mathrm{const}, \; a\in\mathbb{Z}, \; m\in\mathbb{N}$ the following identities hold
    \begin{equation*}
        \begin{split}
            (x-2a)^{2m+1} + 1
            &=\sum_{r=0}^{m} \coeffA{m}{r} \anglePower{x-a}{r} \ast \anglePower{x-a}{r} \\
            (x-2a)^{2m+1} - 1
            &=\sum_{r=0}^{m} \coeffA{m}{r} \curvePower{x-a}{r} \ast \curvePower{x-a}{r}
        \end{split}
    \end{equation*}
    Note that $a$ is parameter of $\anglePower{x-a}{r}$, $\curvePower{x-a}{r}$ by definitions
    \eqref{def_macaulay_bracket}, \eqref{def_macaulay_bracket_strict}.
    By Corollary~\ref{cor_bin_exp_and_macaulay_conv} and Property~\ref{ppty_odd_power_and_iverson},
    for each $s,x,y \in \mathbb{N}$
    \begin{equation*}
        (x+y)^{\iversonBracket} \left(-1 + \sum_{r=0}^{h} \coeffA{h}{r} \anglePower{x+y}{r} \ast \anglePower{x+y}{r} \right)
        \equiv
        \sum_{k=0}^{s} \binom{s}{k} x^{s-k} y^k,
    \end{equation*}
    where $h=\frac{s-1}{2}$.
    In terms of convolution $\curvePower{x}{n} \ast \curvePower{x}{n}$ for $s\geq 1, \; x+y\geq 2$ we also have
    the following relation.
    By Corollary~\ref{cor_bin_exp_and_macaulay_conv_strict} and Property~\ref{ppty_odd_power_and_iverson},
    for each $s,x,y\in\mathbb{N}$
    \begin{equation*}
        (x+y)^{\iversonBracket} \left(1 + \sum_{r=0}^{h} \coeffA{h}{r} \curvePower{x+y}{r} \ast \curvePower{x+y}{r} \right)
        \equiv
        \sum_{k=0}^{s} \binom{s}{k} x^{s-k} y^k,
    \end{equation*}
    where $h=\frac{s-1}{2}$.


    \subsection{Generalisation for Multinomial case}
    In this section we'd like to discus the multinomial cases of Theorems~\ref{thm_odd_power_by_macaulays_convolution},
    ~\ref{thm_odd_power_by_macaulays_convolution_strict}.
    We have the following relation involving Multinomial expansion and discrete convolution of
    power function $\anglePower{x}{n}$
    \begin{cor}
        \label{cor_mult_exp_and_macaulay_conv}
        (Partial case of Theorem~\ref{thm_odd_power_by_macaulays_convolution} for Multinomials.)
        For each $\multifoldSum{t} \geq 1, \; x_1, x_2, \ldots, x_t, m \in \mathbb{N}$
        \begin{equation*}
            \begin{split}
                \sum_{r=0}^{m} \coeffA{m}{r} \anglePower{\multifoldSum{t}}{r} \ast \anglePower{\multifoldSum{t}}{r} \\
                \equiv
                1 + \sum_{\multifoldSum[k]{t}=2m+1} \binom{2m+1}{k_1, k_2,\ldots, k_t} \prod_{\ell=1}^{t} x_\ell^{k_\ell}
            \end{split}
        \end{equation*}
    \end{cor}
    For instance, for $m=1$ and trinomials the corollary~\ref{cor_mult_exp_and_macaulay_conv} gives
    \begin{equation*}
        \begin{split}
            &\sum_{r=0}^{1} \coeffA{1}{r} \anglePower{x+y+z}{r} \ast \anglePower{x+y+z}{r} \\
            &=1 + x + y + z - (x + y + z) (1 + x + y + z) (1 - 3 x - 3 y - 3 z + 2 (x + y + z)) \\
            &=1 + x^3 + 3 x^2 y + 3 x y^2 + y^3 + 3 x^2 z + 6 x y z + 3 y^2 z + 3 x z^2 + 3 y z^2 + z^3
        \end{split}
    \end{equation*}
    \begin{cor}
        \label{cor_mult_exp_and_macaulay_conv_strict}
        (Partial case of Theorem~\ref{thm_odd_power_by_macaulays_convolution_strict} for Multinomials.)
        For each $\multifoldSum{t} \geq 1, \; x_1,x_2,\ldots,x_t,m\in\mathbb{N}$
        \begin{equation*}
            \begin{split}
                \sum_{r=0}^{m} \coeffA{m}{r}
                \curvePower{\multifoldSum{t}}{r} \ast \curvePower{\multifoldSum{t}}{r} \\
                \equiv
                -1 + \sum_{\multifoldSum[k]{t}=2m+1} \binom{2m+1}{k_1, k_2,\ldots, k_t}
                \prod_{\ell=1}^{t} x_\ell^{k_\ell}
            \end{split}
        \end{equation*}
    \end{cor}
    For example, for $m=1$ and trinomials the Corollary~\ref{cor_mult_exp_and_macaulay_conv_strict} gives
    \begin{equation*}
        \begin{split}
            &\sum_{r=0}^{1} \coeffA{1}{r} \anglePower{x+y+z}{r} \ast \anglePower{x+y+z}{r} \\
            &=x + y + z -1 - (x + y + z -1) (x + y + z) (2 (x + y + z) - 1 - 3 x - 3 y - 3 z) \\
            &=x^3 + 3 x^2 y + 3 x y^2 + y^3 + 3 x^2 z + 6 x y z + 3 y^2 z + 3 x z^2 + 3 y z^2 + z^3-1
        \end{split}
    \end{equation*}
    By Corollary~\ref{cor_mult_exp_and_macaulay_conv}, Corollary~\ref{cor_mult_exp_and_macaulay_conv_strict}
    and Property~\ref{ppty_odd_power_and_iverson}, for every $\multifoldSum{t} \geq 1, m\in\mathbb{N}$
    the following relation between discrete convolutions and Multinomial expansion holds
    \begin{equation*}
        \begin{split}
            &(\multifoldSum{t})^s \\
            &=(\multifoldSum{t})^{\iversonBracket}
            \left(-1 + \sum_{k=0}^{h} \coeffA{h}{k}
            \anglePower{\multifoldSum{t}}{k} \ast \anglePower{\multifoldSum{t}}{k} \right) \\
            &=(\multifoldSum{t})^{\iversonBracket}
            \left(1 + \sum_{k=0}^{h} \coeffA{h}{k}
            \curvePower{\multifoldSum{t}}{k} \ast \curvePower{\multifoldSum{t}}{k} \right) \\
            &\equiv \sum_{\multifoldSum[k]{t}=s} \binom{s}{k_1, k_2, \ldots, k_t}
            \prod_{\ell=1}^{t} x_{\ell}^{k_\ell},
        \end{split}
    \end{equation*}
    where $h=\frac{s-1}{2}$.


    \section{Power function as a product of certain matrices}
    Let be a matrices
    \begin{equation*}
        \mathcal{A}_{j} =
        \begin{bmatrix}
            1, & 1, & \ldots & 1
        \end{bmatrix}
        \in \mathbb{N}^{1 \times j}
    \end{equation*}

    \begin{equation*}
        \mathcal{B}_{i,j}(n) =
        \begin{bmatrix}
            \alpha_{1,0} & \alpha_{1,0} & \cdots & \alpha_{1,j} \\
            \alpha_{2,0} & \alpha_{2,0} & \cdots & \alpha_{2,j} \\
            \vdots \\
            \alpha_{i,0} & \alpha_{i,0} & \cdots & \alpha_{i,j} \\
        \end{bmatrix} \in \mathbb{N}^{i \times j},
    \end{equation*}
    where $\alpha_{i,j} = i^j(n-i)^j$.
    \begin{equation*}
        \mathcal{C}_{m}=
        \begin{bmatrix}
            \coeffA{m}{0} \\
            \coeffA{m}{1} \\
            \vdots \\
            \coeffA{m}{m}
        \end{bmatrix} \in \mathbb{R}^{m \times 1}
    \end{equation*}
    Therefore, by Lemma~\ref{lemma_polynomial_p_and_odd_power}, the following identity is true
    \begin{equation*}
        x^{2m+1} = \mathcal{A}_{m} \times \mathcal{B}_{m,x}(x) \times \mathcal{C}_{m}
    \end{equation*}
    For example,
    \begin{equation*}
        4^{2 \cdot 3+1} = [1,1,1,1]
        \cdot
        \begin{bmatrix}
            3^0 & 3^1 & 3^2 & 3^3 \\
            4^0 & 4^1 & 4^2 & 4^3 \\
            3^0 & 3^1 & 3^2 & 3^3 \\
            0^0 & 0^1 & 0^2 & 0^3
        \end{bmatrix}
        \cdot
        \begin{bmatrix}
            1 \\
            -14 \\
            0 \\
            140
        \end{bmatrix}
    \end{equation*}


    \section{Derivation of coefficients \texorpdfstring{$\coeffA{m}{r}$}{A[m,r]}}
    By Lemma~\ref{lemma_polynomial_p_and_odd_power} for every $m \geq 0 \in\mathbb{N}$
    \begin{equation*}
        n^{2m+1} = \polynomialP{m}{n}{n} = \sum_{r=0}^{m} \coeffA{m}{r} \sum_{k=0}^{n-1} k^r (n-k)^r
    \end{equation*}
    The coefficients $\coeffA{m}{r}$ could be evaluated using the binomial expansion of $\sum_{k=0}^{n-1} k^r (n-k)^r$
    \begin{equation*}
        \sum_{k=0}^{n-1} k^r (n-k)^r
        =\sum_{k=0}^{n-1} k^r \sum_{j=0}^{r} (-1)^j \binom{r}{j} n^{r-j} k^{j}
        =\sum_{j=0}^{r} (-1)^j \binom{r}{j} n^{r-j} \sum_{k=0}^{n-1} k^{r+j}
    \end{equation*}
    Using Faulhaber's formula $\sum _{k=1}^{n} k^{p} = \frac{1}{p+1}\sum _{j=0}^{p} \binom{p+1}{j}
    \bernoulli{j} n^{p+1-j}$ we get
    \begin{equation}
        \label{proof1}
        \begin{split}
            \sum_{k=0}^{n-1} k^r (n-k)^r
            &=\sum_{j=0}^{r} \binom{r}{j} n^{r-j} \frac{(-1)^j}{r+j+1}
            \left[\sum_{s} \binom{r+j+1}{s} \bernoulli{s} n^{r+j+1-s} - \bernoulli{r+j+1} \right] \\
            &=\sum_{j,s} \binom{r}{j} \frac{(-1)^j}{r+j+1} \binom{r+j+1}{s} \bernoulli{s} n^{2r+1-s}
            -\sum_{j} \binom{r}{j} \frac{(-1)^j}{r+j+1} \bernoulli{r+j+1} n^{r-j} \\
            &=\sum_{s} \underbrace{\sum_{j} \binom{r}{j} \frac{(-1)^j}{r+j+1} \binom{r+j+1}{s}}_{S(r)}
            \bernoulli{s} n^{2r+1-s} \\
            &-\sum_{j} \binom{r}{j} \frac{(-1)^j}{r+j+1} \bernoulli{r+j+1} n^{r-j}
        \end{split}
    \end{equation}
    where $\bernoulli{s}$ are Bernoulli numbers and $\bernoulli{1}=\frac{1}{2}$.
    Now, we notice that
    \begin{equation*}
        S(r) = \sum_{j} \binom{r}{j} \frac{(-1)^j}{r+j+1} \binom{r+j+1}{s}
        =\begin{cases}
             \frac{1}{(2r+1) \binom{2r}r}, & \text{if } s=0;\\
             \frac{(-1)^r}{s} \binom{r}{2r-s+1}, & \text{if } s>0.
        \end{cases}
    \end{equation*}
    In particular, the last sum is zero for $0<s\leq r$.
    Therefore, expression \eqref{proof1} takes the form
    \begin{equation*}
        \begin{split}
            \sum_{k=0}^{n-1} k^r (n-k)^r
            &=\frac{1}{(2r+1) \binom{2r}{r}} n^{2r+1}
            +\underbrace{\sum_{s \geq 1} \frac{(-1)^r}{s} \binom{r}{2r-s+1} \bernoulli{s} n^{2r+1-s}}_{(\star)} \\
            &-\underbrace{\sum_{j} \binom{r}{j} \frac{(-1)^j}{r+j+1} \bernoulli{r+j+1} n^{r-j}}_{(\diamond)}
        \end{split}
    \end{equation*}
    Hence, introducing $\ell=2r+1-s$ to $(\star)$ and $\ell=r-j$ to $(\diamond)$, we get
    \begin{equation*}
        \begin{split}
            \sum_{k=0}^{n-1} k^r (n-k)^r
            &=\frac{1}{(2r+1) \binom{2r}{r}} n^{2r+1}
            +\sum_{\ell = 2r+1-s} \frac{(-1)^r}{2r+1-\ell} \binom{r}{\ell} \bernoulli{2r+1-\ell} n^{\ell} \\
            &-\sum_{\ell = r-j} \binom{r}{\ell} \frac{(-1)^{j-\ell}}{2r+1-\ell} \bernoulli{2r+1-\ell} n^{\ell}
        \end{split}
    \end{equation*}
    Since that $j-\ell=j-(r-j)=2j-r$
    \begin{equation*}
        \begin{split}
            \sum_{k=0}^{n-1} k^r (n-k)^r
            &=\frac{1}{(2r+1) \binom{2r}{r}} n^{2r+1}
            +(-1)^{r} \sum_{\ell = 2r+1-s} \frac{1}{2r+1-\ell} \binom{r}{\ell} \bernoulli{2r+1-\ell} n^{\ell} \\
            &-\frac{1}{(-1)^{r}} \sum_{\ell = r-j} \binom{r}{\ell} \frac{(-1)^{2j}}{2r+1-\ell} \bernoulli{2r+1-\ell} n^{\ell} \\
            &=\frac{1}{(2r+1) \binom{2r}{r}}n^{2r+1}
            +2 \sum_{\text{odd } \ell = 1,3,\ldots}^{r} \frac{(-1)^r}{2r+1-\ell} \binom{r}{\ell} \bernoulli{2r+1-\ell} n^{\ell}
        \end{split}
    \end{equation*}
    Using the definition of $\coeffA{m}{r}$ coefficients, we obtain the following identity for polynomials in $n$
    \begin{equation}
        \label{proof2}
        \sum_{r=0}^{m} \coeffA{m}{r} \frac{1}{(2r+1) \binom{2r}{r}} n^{2r+1}
        +2 \sum_{r=0}^{m}\sum_{\text{odd } \ell = 1,3,\ldots}^{r} \coeffA{m}{r} \frac{(-1)^r}{2r+1-\ell}
        \binom{r}{\ell} \bernoulli{2r+1-\ell} n^{\ell}
        \equiv
        n^{2m+1}
    \end{equation}
    Taking the coefficient of $n^{2r+1}$ for $r=m$ in \eqref{proof2} we get $\coeffA{m}{m} = (2m+1) \binom{2m}{m}$.
    Since that $\text{odd } \ell \leq r$ in explicit form is $2j + 1 \leq r$, it follows that $j \leq \frac{m-1}{2}$,
    where $j$ is iterator.
    Therefore, taking the coefficient of $n^{2j+1}$ for an integer $j$ in the range $\frac{m}{2} \leq j \leq m$,
    we get $\coeffA{m}{j} = 0$.
    Taking the coefficient of $n^{2d+1}$ for $d$ in the range $m/4 \leq d < m/2$ we get
    \begin{equation*}
        \coeffA{m}{d} \frac{1}{(2d+1) \binom{2d}{d}}
        +2 (2m+1) \binom{2m}{m} \binom{m}{2d+1} \frac{(-1)^m}{2m-2d} \bernoulli{2m-2d} = 0,
    \end{equation*}
    i.e
    \begin{equation*}
        \coeffA{m}{d} = (-1)^{m-1} \frac{(2m+1)!}{d!d!m!(m-2d-1)!} \frac{1}{m-d} \bernoulli{2m-2d}
    \end{equation*}
    Continue similarly we can express $\coeffA{m}{r}$ for each integer $r$ in range $m/2^{s+1}\leq r < m/2^s$
    (iterating consecutively $s=1,2\dots$) via previously determined values of $\coeffA{m}{d}$ as follows
    \begin{equation*}
        \coeffA{m}{r} =
        (2r+1) \binom{2r}{r} \sum_{d=2r+1}^{m} \coeffA{m}{d} \binom{d}{2r+1} \frac{(-1)^{d-1}}{d-r}
        \bernoulli{2d-2r}
    \end{equation*}


    \section{Verification of the results and examples}
    To fulfill our study we provide an opportunity to verify its results by means of Wolfram Mathematica language.
    It is possible to verify the most important results of the manuscript using the Mathematica programs available
    at github repository
    \href{https://github.com/kolosovpetro/arXiv1603.02468-Mathematica-Implementations}{\textsf{arXiv1603.02468 Mathematica Implementations}}.
    Also, we'd like to show why an odd-power identity, namely Lemma~\ref{lemma_polynomial_p_and_odd_power}
    holds by a few examples.
    We arrange in tables the values of $\polynomialL{m}{n}{k}$ to show
    that $\polynomialP{m}{n}{n}=\polynomialL{m}{n}{0}+\polynomialL{m}{n}{1}+\cdots+\polynomialL{m}{n}{n-1}=n^{2m+1}$.
    For example, for $m=1$ we have the following values of $\polynomialL{1}{n}{k}$
    \begin{table}[H]
        \begin{tabular}{c|cccccccc}
            $n/k$ &0 &1 &2 &3 &4 &5 &6 &7 \\[3px]
            \hline
            0 &1 & & & & & & & \\
            1 &1 &1 & & & & & & \\
            2 &1 &7 &1 & & & & & \\
            3 &1 &13 &13 &1 & & & & \\
            4 &1 &19 &25 &19 &1 & & & \\
            5 &1 &25 &37 &37 &25 &1 & & \\
            6 &1 &31 &49 &55 &49 &31 &1 & \\
            7 &1 &37 &61 &73 &73 &61 &37 &1
        \end{tabular}
        \caption{Triangle generated by $\polynomialL{1}{n}{k}, \ 0 \leq k \leq n$.}
            \label{tab_3}
    \end{table}
    From Table~\ref{tab_3} it is seen that
    \begin{equation*}
        \begin{split}
            \polynomialP{1}{0}{0} &= 0 = 0^3 \\
            \polynomialP{1}{1}{1} &= 1 = 1^3 \\
            \polynomialP{1}{2}{2} &= 1+7 = 2^3 \\
            \polynomialP{1}{3}{3} &= 1+13+13 = 3^3 \\
            \polynomialP{1}{4}{4} &= 1+19+25+19 = 4^3 \\
            \polynomialP{1}{5}{5} &= 1+25+37+37+25 = 5^3
        \end{split}
    \end{equation*}
    Another case, for $m=2$ we have the following values of $\polynomialL{2}{n}{k}$
    \begin{table}[H]
        \begin{tabular}{c|cccccccc}
            $n/k$ &0 &1 &2 &3 &4 &5 &6 &7 \\ [3px]
            \hline
            0 &1 & & & & & & & \\
            1 &1 &1 & & & & & & \\
            2 &1 &31 &1 & & & & & \\
            3 &1 &121 &121 &1 & & & & \\
            4 &1 &271 &481 &271 &1 & & & \\
            5 &1 &481 &1081 &1081 &481 &1 & & \\
            6 &1 &751 &1921 &2431 &1921 &751 &1 & \\
            7 &1 &1081 &3001 &4321 &4321 &3001 &1081 &1
        \end{tabular}
        \caption{Triangle generated by $\polynomialL{2}{n}{k}, \ 0\leq k\leq n$.}
        \label{tab_4}
    \end{table}
    Again, an odd-power identity~\ref{lemma_polynomial_p_and_odd_power} holds
    \begin{equation*}
        \begin{split}
            \polynomialP{2}{0}{0} &= 0 = 0^5 \\
            \polynomialP{2}{1}{1} &= 1 = 1^5 \\
            \polynomialP{2}{2}{2} &= 1+31 = 2^5 \\
            \polynomialP{2}{3}{3} &= 1+121+121 = 3^5 \\
            \polynomialP{2}{4}{4} &= 1+271+481+271 = 4^5 \\
            \polynomialP{2}{5}{5} &= 1+481+1081+1081+481 = 5^5
        \end{split}
    \end{equation*}
    Tables ~\ref{tab_3}, ~\ref{tab_4} are entries \href{https://oeis.org/A287326}{\texttt{A287326}}, \href{https://oeis.org/A300656}{\texttt{A300656}} in \cite{Sloane_theencyclopedia}.


    \section{Acknowledgements}
    We would like to thank to Dr. Max Alekseyev (Department of Mathematics and Computational Biology,
    George Washington University) for sufficient help in the derivation of $\coeffA{m}{r}$ coefficients.
    Also, we'd like to thank to OEIS editors Michel Marcus, Peter Luschny, Jon E. Schoenfield and others
    for their patient, faithful volunteer work and for their useful comments and suggestions during the
    editing of the sequences connected with this manuscript.


    \section{Conclusion}
    In this manuscript we have discussed and the $2m+1$-degree integer valued polynomials $\polynomialP{m}{b}{n}$.
    Basing on $\polynomialP{m}{b}{n}$ we established a relation between Binomial theorem and discrete convolution of power function. Also, it is shown that odd binomial expansion is partial case of $\polynomialP{m}{b}{n}$.
    \bibliographystyle{alpha}
    \bibliography{references}
\end{document}
