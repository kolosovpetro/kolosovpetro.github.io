\documentclass[12pt, letterpaper]{amsart}
\usepackage[left=1in,right=1in,bottom=1in,top=1in]{geometry}
\usepackage{amsfonts}
\usepackage{amsmath, amssymb}
\usepackage[font=small,labelfont=bf]{caption}
\usepackage[pdfpagelabels,hyperindex,colorlinks=true,linkcolor=blue,urlcolor=magenta,citecolor=green]{hyperref}
\usepackage{amsthm}
\usepackage{float}
\usepackage{mathrsfs}
\usepackage{colonequals}
\usepackage{xparse}

\newcommand \power [2]{\langle #1 \rangle^{#2}}
\newcommand \iverson[1][s]{[#1 \ \mathrm{is} \ \mathrm{even}]}
\newcommand \floor [1]{\lfloor #1 \rfloor}
\newcommand \coeffA [3][A]{\mathbf{#1}_{{#2},{#3}}}
\newcommand \coeffH [4][H]{\mathbf{#1}_{{#2},{#3}}(#4)}
\newcommand \coeffX [4][X]{\mathbf{#1}_{{#2},{#3}}(#4)}
\newcommand \commonPowSum [3][S]{\mathbf{#1}_{#2}(#3)}
\NewDocumentCommand \mynotation { O{P} E{^_}{mb} D(){n} }{\mathbf{#1}^{#2}_{#3}(#4)}
\NewDocumentCommand \convSum { O{C} E{^_}{rn} D(){b} }{\mathbf{#1}^{#2}_{#3}(#4)}
%%\NewDocumentCommand \coeffH { O{H} E{_}{r} D(){k}}{\mathbf{#1}_{#2}(#3)}
%%\NewDocumentCommand \coeffX { O{X} E{_}{r} D(){b}}{\mathbf{#1}_{#2}(#3)}
\NewDocumentCommand \polynomL { O{L} E{_}{m} D(){n,k}}{\mathbf{#1}_{#2}(#3)}

\newtheorem{thm}{Theorem}[section]
\newtheorem{cor}[thm]{Corollary}
\newtheorem{prop}[thm]{Proposition}
\newtheorem{lem}[thm]{Lemma}
\newtheorem{conj}[thm]{Conjecture}
\newtheorem{quest}[thm]{Question}
\newtheorem{ppty}[thm]{Property}
\newtheorem{ppties}[thm]{Properties}
\newtheorem{claim}[thm]{Claim}

\theoremstyle{definition}
\newtheorem{defn}[thm]{Definition}
\newtheorem{defns}[thm]{Definitions}
\newtheorem{con}[thm]{Construction}
\newtheorem{exmp}[thm]{Example}
\newtheorem{exmps}[thm]{Examples}
\newtheorem{notn}[thm]{Notation}
\newtheorem{notns}[thm]{Notations}
\newtheorem{addm}[thm]{Addendum}
\newtheorem{exer}[thm]{Exercise}
\newtheorem{limit}[thm]{Limitation}

\theoremstyle{remark}
\newtheorem{rem}[thm]{Remark}
\newtheorem{rems}[thm]{Remarks}
\newtheorem{warn}[thm]{Warning}
\newtheorem{sch}[thm]{Scholium}

\makeatletter
\let\c@equation\c@thm
\raggedbottom
\makeatother
\numberwithin{equation}{section}
%--------Meta Data: Fill in your info------
\title{Formulas}
\date{\today}
\hypersetup{
pdftitle={Formulas},
pdfsubject={11},
pdfauthor={Petro Kolosov},
pdfkeywords={11}
}
\begin{document}
\maketitle
\section{Formula 1 - Coefficients A}
\begin{equation*}
\mathbf{A}_{m,r}\colonequals
\begin{cases}
(2r+1)\binom{2r}{r}, & \mathrm{if } \ r=m, \\
(2r+1)\binom{2r}{r} \sum\limits_{d=2r+1}^{m} \mathbf{A}_{m,d} \binom{d}{2r+1} \frac{(-1)^{d-1}}{d-r} B_{2d-2r}, & \mathrm{if } \ 0 \leq r < m, \\
0, & \mathrm{if } \ r<0 \ \mathrm{or } \ r>m.
\end{cases}
\end{equation*}
\begin{equation*}
\coeffA{m}{r} \colonequals
\begin{cases}
(2r+1)\binom{2r}{r}, & \mathrm{if } \ r=m \\
(2r+1)\binom{2r}{r} \sum_{d=2r+1}^{m} \coeffA{m}{d} \binom{d}{2r+1} \frac{(-1)^{d-1}}{d-r} B_{2d-2r}, & \mathrm{if } \ 0 \leq r < m \\
0, & \mathrm{if } \ r<0 \ \mathrm{or } \ r>m
\end{cases}
\end{equation*}
And it is NOT true that A can be defined as

\begin{equation*}
\mathbf{A}_{m,r}\colonequals
\begin{cases}
(2r+1)\binom{2r}{r}, &\quad r=m, \\
(2r+1)\binom{2r}{r} \sum\limits_{\ell} \mathbf{A}_{m,\ell} \binom{\ell}{2r+1} \frac{(-1)^{\ell-1}}{\ell-r} B_{2\ell-2r}[0\leq r< \lfloor m/2 \rfloor], &\quad r\neq m
\end{cases}
\end{equation*}
\section{Formula 2 - Extended form of P}
\begin{equation*}
\begin{split}
\mathbf{P}^{m}_{a,b}(n)
&=\sum\limits_{r=0}^{m}\mathbf{A}_{m,r}\mathbf{Q}^{r}_{a,b}(n)
= \sum\limits_{r=0}^{m}\mathbf{A}_{m,r}(\mathbf{Q}^{r}_{b}(n)-\mathbf{Q}^{r}_{a-1}(n))\\
&=\sum_{t=0}^m\mathbf{X}^{m}_{t}(a,b) (-1)^{m-t} n^t
= \sum_{t=0}^m (\mathbf{X}^{m}_{t}(b)-\mathbf{X}^{m}_{t}(a-1)) (-1)^{m-t} n^t\\
&=\sum_{t=0}^m (-1)^{2m-t} \sum_{k=1}^{2m-t+1} \mathbf{H}_{m,t}(k)((b+1)^k - a^k)n^t
\end{split}
\end{equation*}
\section{Formula 3 - Derivative}
\begin{equation*}
\begin{split}
\frac{\mathrm{d}}{\mathrm{d}t}\mathbf{P}^{(m)}_{i,j}(t)
&=\lim\limits_{t \to u}\frac{\mathbf{P}^{(m)}_{i,j}(t)-\mathbf{P}^{(m)}_{i,j}(u)}{t-u}\\
&=\lim\limits_{t \to u}\frac{1}{t-u}\sum_{r=0}^{m} \mathbf{A}_{m,r}(\mathbf{Q}^{(r)}_{i,j}(t)-\mathbf{Q}^{(r)}_{i,j}(u))
\end{split}
\end{equation*}
\section{Formula 3 - Odd finite differences}
\begin{equation*}
\begin{split}
\Delta_h \mathbf{P}^{(m)}_{i,j}(a)
&= \mathbf{P}^{(m)}_{i,j}(a)-\mathbf{P}^{(m)}_{i,j}(b)\\
&=\left(\sum\limits_{r=0}^{m}\mathbf{A}_{m,r}\mathbf{Q}^{(r)}_{i,j}(a)\right)-\left(\sum\limits_{r=0}^{m}\mathbf{A}_{m,r}\mathbf{Q}^{(r)}_{i,j}(b)\right)\\
&=\sum_{r=0}^{m} \mathbf{A}_{m,r}(\mathbf{Q}^{(r)}_{i,j}(a)-\mathbf{Q}^{(r)}_{i,j}(b)), \quad b\leq a
\end{split}
\end{equation*}
\section{Formula 4 - Even finite differences}
\begin{equation*}
\begin{split}
\Delta_h(x^{2m})
&=(x+h)^{2m}-x^{2m}=\frac{(x+h)^{2m+1}}{x+h}-\frac{x^{2m+1}}{x}=\frac{x(x+h)^{2m+1}-(x+h)x^{2m+1}}{x(x+h)}\\
&=\frac{x(x+h)^{2m+1}-xx^{2m+1}-hx^{2m+1}}{x(x+h)}=\frac{(x+h)^{2m+1}-x^{2m+1}-hx^{2m}}{x+h}\\
&=\frac{\Delta_h(x^{2m+1})-hx^{2m}}{x+h}
\end{split}
\end{equation*}
\section{Formula 5 - r-fold power sum of odds}
\begin{equation*}
\sum_{n=0}^s n^{2m+1}=\sum_{n=0}^s\mathbf{X}_m(n)=\sum_{n=0}^s\sum_{r=0}^{m} \mathbf{A}_{m,r} \mathbf{Q}_r(n)=\sum_{r=0}^{m} \mathbf{A}_{m,r} \sum_{n=0}^s\mathbf{Q}_r(n)
\end{equation*}
\section{Formula 6 - Binomial coefficients as polynomials}
\begin{equation*}
\binom{t}{k}=\frac{t(t-1)(t-2)\cdots(t-k+1)}{k(k-1)(k-2)\cdots 2\cdot 1}=\frac{1}{k!}\prod_{w=0}^{k-1}(t-w)
\end{equation*}
\section{Formula 7 - Binomial expansion and L(m,n,k) identity}
\begin{equation*}
\mathbf{P}^{(m)}_{0,a+n-1}(a+n) \equiv (a+n)\sum_{k=0}^{2m}\binom{2m}{k}a^{2m-k}n^k
\end{equation*}
\section{Formula 8 - Limits of sums}
\begin{equation*}
\lim\limits_{t\to n} \mathbf{P}_m(n,t) = n^{2m+1}
\end{equation*}
\section{Formula 9 - Symmetry of Lm(n,k)}
\begin{equation*}
\mathbf{L}_m(n,k)=\mathbf{L}_m(n,n-k)
\end{equation*}
\section{Formula 10 - Relations between P, A, Q (new notation)}
\begin{equation*}
\mathbf{P}^{(m)}_{i,j}(n)
=\sum\limits_{k=i}^{j}\mathbf{L}_m(n,k)
=\sum\limits_{r=0}^{m}\mathbf{A}_{m,r}\mathbf{Q}^{(r)}_{i,j}(n)
=\sum\limits_{r=0}^{m}\mathbf{A}_{m,r}\sum\limits_{k=i}^{j}\mathbf{U}^r(n,k)
\end{equation*}
\section{Formula 11 - Odd power identities}
\begin{equation*}
n^{2m+1}=\sum\limits_{r=0}^{m}\mathbf{A}_{m,r}\mathbf{Q}^{(r)}_{0,n-1}(n)
\end{equation*}
\begin{equation*}
n^{2m+1}=\sum\limits_{r=0}^{m}\mathbf{A}_{m,r}\mathbf{Q}^{(r)}_{1,n}(n)
\end{equation*}
\begin{equation*}
n^{2m-1}=\sum\limits_{r=0}^{m-1}\mathbf{A}_{m-1,r}\mathbf{Q}^{(r)}_{1,n}(n)
\end{equation*}
\begin{equation*}
\begin{split}
n^s
&= \frac{1}{n^{\delta_{1, s\bmod2}}}\sum_{r=0}^{\lfloor s/2\rfloor}\mathbf{A}_{\lfloor s/2\rfloor,r}\mathbf{Q}^{(r)}_{1,n}(n)\\
&= \frac{1}{n^{\delta_{s\bmod2}}}\sum_{r=0}^{\lfloor s/2\rfloor}\mathbf{A}_{\lfloor s/2\rfloor,r}\mathbf{Q}^{(r)}_{1,n}(n) \\
&= n^{\delta_{s\bmod2}}\sum_{r=0}^{\lfloor (s-1)/2\rfloor}\mathbf{A}_{\lfloor (s-1)/2\rfloor,r}\mathbf{Q}^{(r)}_{1,n}(n)
\end{split}
\end{equation*}
\begin{equation*}
\begin{split}
n^s
&= \frac{1}{n^{\delta_{1, s\bmod2}}}\mathbf{P}^{\lfloor s/2\rfloor}_{1,n}(n)\\
&= \frac{1}{n^{\delta_{s\bmod2}}}\sum_{r=0}^{\lfloor s/2\rfloor}\mathbf{A}_{\lfloor s/2\rfloor,r}\mathbf{Q}^{(r)}_{1,n}(n) \\
&= n^{\delta_{s\bmod2}}\sum_{r=0}^{\lfloor (s-1)/2\rfloor}\mathbf{A}_{\lfloor (s-1)/2\rfloor,r}\mathbf{Q}^{(r)}_{1,n}(n)
\end{split}
\end{equation*}
\section{Formula 12 - Other properties of odd power identities}
\begin{equation*}
(2s+1)^{2m+1}+1=2\sum_{k=0}^{s} \mathbf{L}_m(2s+1,k)
\end{equation*}
\begin{equation*}
(2s+2)^{2m+1}+1=\mathbf{L}_m(2s+2,s+1)+2\sum_{k=0}^{s} \mathbf{L}_m(2s+2,k)
\end{equation*}
\begin{equation*}
n^{2m+1} = \iverson[n] \polynomL(n, n/2) + \sum_{k=0}^{\floor{(n-1)/2}} \polynomL
\end{equation*}
\section{Formula 13 - Polynomials P(m,n,i,j) complete form}
\begin{equation*}
\begin{split}
\mathbf{P}^{(1)}_{i,j}(n)
&=1 \\
&-3 i^2 + 2 i^3 \\
&-3 j^2 - 2 j^3 \\
&+3 i n - 3 i^2 n \\
&+3 j n + 3 j^2 n
\end{split}
\end{equation*}

\begin{equation*}
\begin{split}
\mathbf{P}^{(2)}_{i,j}(n) =
1 &- 10 i^3 + 15 i^4 - 6 i^5\\
  &+ 10 j^3 + 15 j^4 + 6 j^5 \\
  &+ 15 i^2 n - 30 i^3 n + 15 i^4 n\\
  &- 15 j^2 n - 30 j^3 n - 15 j^4 n \\
  &- 5 i n^2 + 15 i^2 n^2 - 10 i^3 n^2 \\
  &+ 5 j n^2 + 15 j^2 n^2 + 10 j^3 n^2
\end{split}
\end{equation*}

\begin{equation*}
\begin{split}
\mathbf{P}^{(3)}_{i,j}(n) &=1 \\
&+7 i^2 - 28 i^3 + 70 i^5 - 70 i^6 + 20 i^7 \\
&+7 j^2 + 28 j^3 -70 j^5 - 70 j^6 - 20 j^7 \\
&-7 i n + 42 i^2 n - 175 i^4 n + 210 i^5 n - 70 i^6 n \\
&-7 j n - 42 j^2 n + 175 j^4 n + 210 j^5 n +70 j^6 n \\
&-14 i n^2 + 140 i^3 n^2 - 210 i^4 n^2 + 84 i^5 n^2 \\
&+14 j n^2 - 140 j^3 n^2 - 210 j^4 n^2 - 84 j^5 n^2 \\
&-35 i^2 n^3 + 70 i^3 n^3 - 35 i^4 n^3 \\
&+35 j^2 n^3 + 70 j^3 n^3 + 35 j^4 n^3
\end{split}
\end{equation*}

\begin{equation*}
\begin{split}
\mathbf{P}^{(4)}_{i,j}(n)
&=1 \\
&+60 i^2 - 180 i^3 + 294 i^5 - 420 i^7 + 315 i^8 - 70 i^9 \\
&+60 j^2 + 180 j^3 - 294 j^5 + 420 j^7 + 315 j^8 + 70 j^9 \\
&-60 i n +270 i^2 n - 735 i^4 n + 1470 i^6 n - 1260 i^7 n + 315 i^8 n \\
&-60 j n - 270 j^2 n + 735 j^4 n - 1470 j^6 n - 1260 j^7 n -315 j^8 n \\
&-90 i n^2 + 630 i^3 n^2 - 1890 i^5 n^2 + 1890 i^6 n^2 -540 i^7 n^2 \\
&+90 j n^2 - 630 j^3 n^2 + 1890 j^5 n^2 + 1890 j^6 n^2 +540 j^7 n^2 \\
&-210 i^2 n^3 + 1050 i^4 n^3 - 1260 i^5 n^3 +420 i^6 n^3 \\
&+210 j^2 n^3 - 1050 j^4 n^3 - 1260 j^5 n^3 -420 j^6 n^3 \\
&+21 i n^4 - 210 i^3 n^4 + 315 i^4 n^4 - 126 i^5 n^4 \\
&-21 j n^4 + 210 j^3 n^4 + 315 j^4 n^4 + 126 j^5 n^4
\end{split}
\end{equation*}
\section{Formula 14 - Another Form}
\begin{equation*}
T\strut^{2s+1}=\sum_{r=0}^{s}\sum_{\kappa=1}^{2s-r+1} (-1)\strut^{2s-r} \mathscr{L}_{s,r}(\kappa) \cdot T\strut^{\kappa+r}
\end{equation*}
\section{Formula 15 - Why polynomials P(m,n,i,j) in i,j,n ?}
\begin{equation*}
\begin{split}
\sum_{k=a}^{b}\sum_{j=0}^m A_{m,j}k^j(n-k)^j
&=\sum_{k=a}^{b}\sum_{j=0}^m A_{m,j}k^j\sum_{t=0}^j\binom{j}{t}n^t(-1)^{j-t}k^{j-t}\\
&=\sum_{t=0}^m n^t \sum_{k=a}^{b}\sum_{j=t}^m \binom{j}{t}A_{m,j}k^{2j-t}(-1)^{j-t}\\
&=\sum_{t=0}^m n^t \sum_{j=t}^m (-1)^{j-t}\binom{j}{t}A_{m,j}\sum_{k=a}^{b}k^{2j-t}
\end{split}
\end{equation*}
\section{Formulas 16 - Alekseyev's approach on U coefficients}
Originally from https://mathoverflow.net/q/304130/113033. (truth)
\begin{equation*}
\begin{split}
\mathbf{P}^{(m)}_{0,T}(n)
&=\sum_{k=0}^T\sum_{j=0}^m A_{m,j}k^j(n-k)^j = \sum_{j=0}^m(-1)^{m-j}\mathbf{X}^{(m)}_{0,T}(j)\cdot n^j \\
&=\sum_{k=0}^T\sum_{j=0}^m A_{m,j}k^j\sum_{t=0}^j\binom{j}{t}n^t(-1)^{t-t}k^{j-t}\\
&=\sum_{t=0}^m n^t \sum_{k=0}^T\sum_{j=t}^m \binom{j}{t}A_{m,j}k^{2j-t}(-1)^{j-t}.
\end{split}
\end{equation*}
Now, taking the coefficient of $n^t$ in above gives (truth):
\begin{equation*}
\mathbf{X}^{(m)}_{0,T}(t) =(-1)^m \sum_{k=0}^T\sum_{j=t}^m \binom{j}{t}A_{m,j}k^{2j-t}(-1)^j, \quad 0\leq t \leq m
\end{equation*}
From this formula it may be not immediately clear why $U_m(T,t)$ represent polynomials in $T$. However, this can be seen if we change the summation order again and use Faulhaber's formula to obtain:
\begin{equation*}
\mathbf{X}^{(m)}_{0,T}(t) = (-1)^m \sum_{j=t}^m \binom{j}{t}A_{m,j} \frac{(-1)^j}{2j-t+1}\sum_{\ell=0}^{2j-t} \binom{2j-t+1}{\ell}B_{\ell}T^{2j-t+1-\ell}.
\end{equation*}
Introducing $k=2j-t+1-\ell$, we further get the formula:
\begin{equation*}
\begin{split}
\mathbf{X}^{(m)}_{0,T}(t)
&=(-1)^m \sum_{k=1}^{2m-t+1} T^k \underbrace{\sum_{j=t}^m \binom{j}{t}A_{m,j} \frac{(-1)^j}{2j-t+1}\binom{2j-t+1}{k}B_{2j-t+1-k}}_{\mathbf{H}_{m,t}(k)}\\
&=(-1)^m \sum_{k=1}^{2m-t+1} \mathbf{H}_{m,t}(k) (T+1)^k
\end{split}
\end{equation*}
Then polynomial $\mathbf{P}^{(m)}_{0,T}(n)$ can be expressed as
\begin{equation*}
\begin{split}
\mathbf{P}^{(m)}_{0,T}(n)
&=\sum_{j=0}^m(-1)^{m-j}(-1)^m \sum_{k=1}^{2m-j+1} \mathbf{H}_{m,j}(k) T^k\cdot n^j \\
&=\sum_{j=0}^m\sum_{k=1}^{2m-j+1} (-1)^{2m-j} \mathbf{H}_{m,j}(k) T^k\cdot n^j
\end{split}
\end{equation*}
In general, for $\mathbf{P}^{(m)}_{a,b}(n)$ we have
\begin{equation*}
\begin{split}
\mathbf{P}^{(m)}_{a,b}(n)
&=\sum_{k=0}^{b}\mathbf{L}_m(n,k)-\sum_{k=0}^{a-1}\mathbf{L}_m(n,k)\\
&=\sum_{j=0}^m\sum_{k=1}^{2m-j+1} (-1)^{2m-j} \mathbf{H}_{m,j}(k) b^k\cdot n^j - \sum_{j=0}^m\sum_{k=1}^{2m-j+1} (-1)^{2m-j} \mathbf{H}_{m,j}(k) (a-1)^k\cdot n^j\\
&=\sum_{j=0}^m (-1)^{2m-j} \sum_{k=1}^{2m-j+1} \mathbf{H}_{m,j}(k)\cdot n^j (b^k - (a-1)^k)\\
&=\sum_{j=0}^m (-1)^{2m-j} \sum_{k=1}^{2m-j+1} \mathbf{H}_{m,j}(k)\cdot n^j \Delta_h (b^k), \quad h=b-a+1
\end{split}
\end{equation*}
\begin{equation*}
\begin{split}
\Delta_h (b^k)=(b+h)^k - b^k
&=\left(\sum_{s=0}^k\binom{k}{s}b^{k-s}h^s\right)-b^k\\
&=b^k+\left(\sum_{s=1}^k\binom{k}{s}b^{k-s}h^s\right)-b^k\\
&=\sum_{s=1}^k\binom{k}{s}b^{k-s}h^s
\end{split}
\end{equation*}
Let be $h=b-a+1$, thus
\begin{equation*}
\begin{split}
\Delta_h (b^k)=(b+h)^k - b^k
&=\sum_{s=1}^k\binom{k}{s}b^{k-s}h^s=\sum_{s=1}^k\binom{k}{s}b^{k-s}(b-a+1)^s \\
&=\sum_{s=1}^k\binom{k}{s}b^{k-s}h^s=\sum_{s=1}^k\binom{k}{s}b^{k-s}\sum_{k_1+k^2+k^3=s} \binom{s}{k_1, k_2, k_3}b^{k_1}(-a)^{k_2}\\
&=\sum_{s=1}^k\binom{k}{s}b^{k-s}\sum_{k_1+k^2+k^3=s}(-1)^{k_2} \binom{s}{k_1, k_2, k_3}b^{k_1}a^{k_2}
\end{split}
\end{equation*}
\begin{equation*}
\begin{split}
\mathbf{P}^{(m)}_{a,b}(n)
&=\sum_{j=0}^m (-1)^{2m-j} \sum_{k=1}^{2m-j+1} \mathbf{H}_{m,j}(k)\cdot n^j (b^k - (a-1)^k) \\
&=\sum_{j=0}^m (-1)^{2m-j} \sum_{k=1}^{2m-j+1} \mathbf{H}_{m,j}(k)(n^jb^k - n^j(a-1)^k)
\end{split}
\end{equation*}
\section{Formulas 17 - Odd power and convolution identities}
\begin{equation*}
n^{2m+1}+1 = \sum_{r\geq 0} \mathbf{A}_{m,r} (f_{r} \ast f_{r})[n], \quad n>0, \quad n\in\mathbb{N}
\end{equation*}
\begin{equation*}
n^{2m+1} - 1 = \sum_{r\geq 0} \mathbf{A}_{m,r} (f_{r,1} \ast f_{r,1})[n], \quad n>0, \quad n\in\mathbb{N}
\end{equation*}
\section{Formula 18 - Definition of H}
$\mathbf{H}_{m,t}(k)$ is real coefficient defined in terms of Bernoulli numbers, Binomial coefficients and $\mathbf{A}_{m,r}$
\begin{equation}
\mathbf{H}_{m,t}(k)\colonequals \sum\limits_{j=t}^m \binom{j}{t}\mathbf{A}_{m,j} \frac{(-1)^j}{2j-t+1}\binom{2j-t+1}{k}B_{2j-t+1-k}, \quad 0\leq t \leq m.
\end{equation}
\section{Formula 19 - Definition of X}
$\mathbf{X}^{m}_{a,b}(t)$ is $2m-t$ degree polynomial in $a,b$ defined involving $\mathbf{A}_{m,r}$
\begin{equation}
\mathbf{X}^{m}_{a,b}(t)\colonequals (-1)^m \sum\limits_{k=a}^b\sum\limits_{j=t}^m \mathbf{A}_{m,j} (-1)^j \binom{j}{t} k^{2j-t}, \quad 0\leq t \leq m
\end{equation}
\section{Experiment on convolution and Iverson's}
\begin{equation*}
(f\ast f)[n]=\sum_{k}f[k]f[n-k].
\end{equation*}

\begin{equation*}
\begin{split}
n_{\geq t}^r \ast n_{\geq t}^r
&= \sum_{k}k_{\geq t}^r(n-k)_{\geq t}^r = \sum_{k}k^r[k\geq t](n-k)^r[n-k\geq t] = \sum_{k}k^r(n-k)^r[k\geq t][n-k\geq t] \\
&= \sum_{k}k^r(n-k)^r[t\leq k \leq n-t]
\end{split}
\end{equation*}
\section{Power sum and Iversons notation}
\begin{equation*}
\begin{split}
\sum_{k=0}^{b} k^r(n-k)^r
&= \sum_{k} k^r(n-k)^r [0\leq k \leq b] = \sum_{k} k^r \sum_{t=0}^{r}(-1)^t \binom{r}{t} n^{r-t}k^t [0\leq k \leq b] \\
&= \sum_{k} k^r \sum_{t}(-1)^t \binom{r}{t} n^{r-t}k^t [0\leq k \leq b] \\
&= \sum_{k}\sum_{t}(-1)^t \binom{r}{t} n^{r-t}k^{2t} [0\leq k \leq b] \\
&= \sum_{t}(-1)^t \binom{r}{t} n^{r-t}\sum_{k}k^{2t} [0\leq k \leq b]
\end{split}
\end{equation*}
\begin{equation*}
\begin{split}
&= \sum_{t}\binom{r}{t} n^{r-t}\sum_{j=0}^{2t}\frac{(-1)^t}{2t+1} \binom{2t+1}{j} B_j b^{r+t+1-j} \\
&= \sum_{t} \binom{r}{t} n^{r-t}\sum_{j}\frac{(-1)^t}{r+t+1} \binom{r+t+1}{j} B_j b^{r+t+1-j} [0\leq j\leq r+t]
\end{split}
\end{equation*}
\section{P involving Iversons}
\section{Modification on X}
\begin{equation*}
\begin{split}
\mathbf{X}^{m}_{t}(a,b)
&= (-1)^m \sum\limits_{j} \mathbf{A}_{m,j} (-1)^j \binom{j}{t} \sum\limits_{k}k^{2j-t}[a\leq k<b][j\geq t] \\
&= (-1)^m \sum\limits_{j} \mathbf{A}_{m,j} (-1)^j \binom{j}{t}\left( \sum\limits_{k=0}^{b}k^{2j-t}-\sum\limits_{k=0}^{a-1}k^{2j-t}\right)[j\geq t] \\
&=\sum\limits_{j\geq t}(-1)^{j+m} \mathbf{A}_{m,j} \binom{j}{t}(S_{2j-t}(b)-S_{2j-t}(a))
\end{split}
\end{equation*}
\section{Binomials and convolution}
\begin{equation*}
\sum_{r} \binom{2m+1}{r} a^{2m+1-r} b^r \equiv -1 + \sum\limits_{r}\mathbf{A}_{m,r} (f_{0}^{r} \ast f_{0}^{r})[a+b]
\end{equation*}
\begin{equation*}
\sum_{r} \binom{2m+1}{r} a^{2m+1-r} b^r \equiv 1 + \sum\limits_{r}\mathbf{A}_{m,r} (f_{1}^{r} \ast f_{1}^{r})[a+b]
\end{equation*}
\section{Convolutional derivative}
\begin{equation*}
\begin{split}
(a+b)^{2m+1}=\sum_{r} \binom{2m+1}{r} a^{2m+1-r} b^r
&\equiv -1+\mathbf{P}^{m}_{a+b+1}(a+b) \\
&= -1+\sum_{r}\mathbf{A}_{m,r}\mathbf{Q}^{r}_{a+b+1}(a+b)\\
&= -1+\sum\limits_{r}\mathbf{A}_{m,r} (f^{r} \ast f^{r})[a+b].
\end{split}
\end{equation*}
\begin{equation*}
\begin{split}
\Delta(x^{2m+1})
&=\sum\limits_{r}\mathbf{A}_{m,r} (f^{r} \ast f^{r})[x+\Delta x]-\sum\limits_{r}\mathbf{A}_{m,r} (f^{r} \ast f^{r})[x] \\
&=\sum\limits_{r}\mathbf{A}_{m,r} \{(f^{r} \ast f^{r})[x+\Delta x] - (f^{r} \ast f^{r})[x]\} \\
&=\sum\limits_{r}\mathbf{A}_{m,r} \Delta (f^{r} \ast f^{r})[x]
\end{split}
\end{equation*}
\begin{equation*}
\begin{split}
\frac{d}{dx}x^{2m+1}
&=\lim\limits_{\Delta x\to0} \sum\limits_{r}\mathbf{A}_{m,r} \frac{\Delta (f^{r} \ast f^{r})[x]}{\Delta x} \\
&=\sum\limits_{r}\mathbf{A}_{m,r} \lim\limits_{\Delta x\to0}\frac{\Delta (f^{r} \ast f^{r})[x]}{\Delta x} \\
&=\sum\limits_{r}\mathbf{A}_{m,r} \frac{d}{dx} (f^{r} \ast f^{r})[x]=\sum\limits_{r}\mathbf{A}_{m,r} (\tfrac{df^{r}}{dx} \ast f^{r})[x]\\
&=\sum\limits_{r}\mathbf{A}_{m,r} r(f^{r-1} \ast f^{r})[x]
\end{split}
\end{equation*}
\end{document}
