\documentclass[12pt, letterpaper]{amsart}
\usepackage[left=1in,right=1in,bottom=1in,top=1in]{geometry}
\usepackage{amsfonts}
\usepackage{amsmath, amssymb}
\usepackage[font=small,labelfont=bf]{caption}
\usepackage[pdfpagelabels,hyperindex,colorlinks=true,linkcolor=blue,urlcolor=magenta,citecolor=green]{hyperref}
\usepackage{amsthm}
\usepackage{float}
\usepackage{mathrsfs}
\usepackage{colonequals}

\newenvironment{myitemize}
{ \begin{itemize}
    \setlength{\itemsep}{4pt}
    \setlength{\parskip}{4pt}
    \setlength{\parsep}{4pt}     }
{ \end{itemize}                  }


\newtheorem{thm}{Theorem}[section]
\newtheorem{cor}[thm]{Corollary}
\newtheorem{prop}[thm]{Proposition}
\newtheorem{lem}[thm]{Lemma}
\newtheorem{conj}[thm]{Conjecture}
\newtheorem{quest}[thm]{Question}
\newtheorem{ppty}[thm]{Property}
\newtheorem{ppties}[thm]{Properties}
\newtheorem{claim}[thm]{Claim}

\theoremstyle{definition}
\newtheorem{defn}[thm]{Definition}
\newtheorem{defns}[thm]{Definitions}
\newtheorem{con}[thm]{Construction}
\newtheorem{exmp}[thm]{Example}
\newtheorem{exmps}[thm]{Examples}
\newtheorem{notn}[thm]{Notation}
\newtheorem{notns}[thm]{Notations}
\newtheorem{addm}[thm]{Addendum}
\newtheorem{exer}[thm]{Exercise}
\newtheorem{limit}[thm]{Limitation}

\theoremstyle{remark}
\newtheorem{rem}[thm]{Remark}
\newtheorem{rems}[thm]{Remarks}
\newtheorem{warn}[thm]{Warning}
\newtheorem{sch}[thm]{Scholium}

\makeatletter
\let\c@equation\c@thm
\raggedbottom
\makeatother
\numberwithin{equation}{section}
%--------Meta Data: Fill in your info------
\title{A relation between Binomial Theorem and Discrete Convolution of Power Function}
\author[Petro Kolosov]{Petro Kolosov}
\email{kolosovp94@gmail.com}
\keywords{Binomial theorem, Convolution, Power function, Multinomial theorem, Binomial coefficient, Multinomial coefficient, Power Sums}
\urladdr{https://kolosovpetro.github.io}
\subjclass[2010]{44A35 (primary), 11C08 (secondary)}
\date{\today}
\hypersetup{
pdftitle={A relation between Binomial Theorem and Discrete Convolution of Power Function},
pdfsubject={Mathematics, Combinatorics, Number theory},
pdfauthor={Petro Kolosov},
pdfkeywords={Binomial theorem, Convolution, Power function, Multinomial theorem, Binomial coefficient, Multinomial coefficient, Power Sums}
}
\usepackage{microtype}
\begin{document}
\begin{abstract}
In this manuscript a relation between Binomial theorem and discrete convolution of the power function $f^r(n)=n^r, \ n\in\mathbb{N}$ is established. It is shown that Binomial expansion of $s$-powered sum $(a+b)^s, \ s>0$ is equivalent to the sum of consequent convolutions of $f^r(n), \ 0\leq r < s$ multiplied by the certain real coefficients. In addition, the relation between Binomial theorem and convolution of $f^r(n)$ is generalised to Multinomial case.
\end{abstract}
\maketitle
\tableofcontents
\section{Introduction}
In this paper we reveal a relation between famous Binomial theorem \cite{AbraSteg72} and discrete convolution of power function $f^r(n)=n^r, \ n\in\mathbb{N}$. The content of the manuscript reaches the main aim of the work through the following milestones. Firstly, we perform a detailed discussion on $2m$-degree integer-valued polynomials $\mathbf{P}^{m}_{a,b}(n)$, see \eqref{p_definition}. We show all the implicit forms of the polynomials $\mathbf{P}^{m}_{a,b}(n)$ and discuss their main properties. Finally, for the first milestone, we arrive to the identity between odd-powered Binomial (and Multinomial) expansions and partial case of $\mathbf{P}^{m}_{a,b}(n)$. As next step, we establish a relation between the polynomials $\mathbf{P}^{m}_{a,b}(n)$ and discrete convolution of the power function $f^r(n)=n^r, \ n\in\mathbb{N}$. This relation is consequence of the following claims:
\begin{itemize}
  \item $\mathbf{P}^{m}_{a,b}(n)$ is in relation with the power sum $\mathbf{Q}^{r}_{b}(n)$, see \eqref{q_definition} for $\mathbf{Q}^{r}_{b}(n)$.
  \item Discrete convolution of power function $f^r(n)=n^r, \ n\in\mathbb{N}$ is partial case of the power sum $\mathbf{Q}^{r}_{b}(n)$.
  \item Polynomials $\mathbf{P}^{m}_{a,b}(n)$ are in relation with the discrete convolution of power function $f^r(n)=n^r, \ n\in\mathbb{N}$.
\end{itemize}
Then, in the subsection (3.1) we particularise obtained results to show the relation between Binomial (and Multinomial) theorem and the discrete convolution of piecewise defined power function.
\subsection{Notation and conventions}
We now set the following notation, which remains fixed for the remainder of this paper:
\begin{myitemize}
\item We strongly believe to D. Knuth's words in \cite{Knuth92}
\begin{quote}
\textit{I  realized long ago that "boundary conditions" on indices of summation are often a handicap and a waste of time.}
\end{quote}
For example, instead of writing
\begin{equation*}
2^n=\sum_{k=0}^n \binom{n}{k}
\end{equation*}
it is much better to write
\begin{equation*}
2^n=\sum_{k} \binom{n}{k}
\end{equation*}
the sum now extends over all integers $k$, but only finitely many terms are nonzero.
\item We believe to \cite{Grah94SN} that exponential function $0^x$ should be defined for all $x$  as
\begin{equation*}
0^x = 1.
\end{equation*}
\item $[P(k)]$ is the Iverson's convention \cite{APL}, where $P(k)$ is logical sentence depending on $k$
\begin{equation*}
[P(k)] =
\begin{cases}
1, &\quad P(k) \ is \ true, \\
0, &\quad otherwise
\end{cases}
\end{equation*}
\item $f^{r}(n)$ is a power function defined on the set $\mathbb{N}$ of natural numbers
\begin{equation*}
f^{r}(n) \colonequals n^r, \quad n\in\mathbb{N}.
\end{equation*}
We assume that set of natural numbers $\mathbb{N}$ starts from zero.
\item $(f\ast f)[n]$ is discrete convolution transform \cite{DiscConv} of the real defined on the set $\mathbb{Z}$ of integers function $f(n)$
\begin{equation*}
(f\ast f)[n]=\sum_{k}f[k]f[n-k].
\end{equation*}
\item $\mathbf{A}_{m,r}$ is a real coefficient defined recursively as
\begin{equation*}
\mathbf{A}_{m,r}\colonequals
\begin{cases}
(2r+1)\binom{2r}{r}, & \mathrm{if } \ r=m, \\
(2r+1)\binom{2r}{r} \sum\limits_{d=2r+1}^{m} \mathbf{A}_{m,d} \binom{d}{2r+1} \frac{(-1)^{d-1}}{d-r} B_{2d-2r}, & \mathrm{if } \ 0 \leq r < m, \\
0, & \mathrm{if } \ r<0 \ \mathrm{or } \ r>m.
\end{cases}
\end{equation*}
where $B_t$ are Bernoulli numbers. We assume that $B_1=\tfrac12$. For $m\geq 11$ the $\mathbf{A}_{m,r}$ takes the fractional values for certain $r$.
\item $\mathbf{Q}^{r}_{a,b}(n)$ is the power sum defined as
\begin{equation}
\label{q_definition}
\mathbf{Q}^{r}_{a,b}(n)\colonequals\sum\limits_{a\leq k<b} k^r(n-k)^r, \quad (n,r)\in\mathbb{Z}.
\end{equation}
Notation $\mathbf{Q}^{r}_{b}(n)$ is an equivalent to $\mathbf{Q}^{r}_{a,b}(n)$ with set $a=0$, i.e $\mathbf{Q}^{r}_{b}(n) \equiv \mathbf{Q}^{r}_{0,b}(n)$.
\item $S_p(n)$ is a common power sum
\begin{equation*}
S_p(n) = \sum_{0\leq k < n} k^p.
\end{equation*}
\item $\mathbf{L}_m(n,k)$ are polynomials of degree $2m$ in $n,k$ defined involving coefficients $\mathbf{A}_{m,r}$
\begin{equation*}
\mathbf{L}_m(n,k)\colonequals\sum\limits_{r} \mathbf{A}_{m,r} k^r(n-k)^r, \quad m\in\mathbb{N}, \quad (n,k)\in\mathbb{Z}.
\end{equation*}
\item $\mathbf{P}^{m}_{a,b}(n)$ are polynomials of degree $2m$ in $a, b, n$. Polynomials $\mathbf{P}^{m}_{a,b}(n)$ are defined as a sum of $\mathbf{L}_m(n,k)$ over $k$,
\begin{equation}
\label{p_definition}
\mathbf{P}^{m}_{a,b}(n)\colonequals \sum\limits_{a\leq k<b} \mathbf{L}_{m}(n,k), \quad n\in\mathbb{Z}.
\end{equation}
Notation $\mathbf{P}^{m}_{b}(n)$ is an equivalent to $\mathbf{P}^{m}_{a,b}(n)$ with set $a=0$, i.e $\mathbf{P}^{m}_{b}(n) \equiv \mathbf{P}^{m}_{0,b}(n)$.
\item $\mathbf{X}^{m}_{t}(a,b)$ are polynomials of degree $2m-t$ in $a,b$ defined as
\begin{equation*}
\mathbf{X}^{m}_{t}(a,b)\colonequals (-1)^m \sum\limits_{j\geq t} \mathbf{A}_{m,j} (-1)^j \binom{j}{t} \sum\limits_{a\leq k<b}k^{2j-t}, \quad 0\leq t\leq m.
\end{equation*}
Notation $\mathbf{X}^{m}_{t}(b)$ is an equivalent to $\mathbf{X}^{m}_{t}(a,b)$ with set $a=0$, i.e $\mathbf{X}^{m}_{t}(b) \equiv \mathbf{X}^{m}_{t}(0,b)$.
\item $\mathbf{H}_{m,t}(k)$ are real coefficients defined in terms of Bernoulli numbers $B_t$, Binomial coefficients $\binom{t}{k}$ and $\mathbf{A}_{m,r}$
\begin{equation*}
\mathbf{H}_{m,t}(k)\colonequals \sum\limits_{j\geq t}\binom{j}{t}\mathbf{A}_{m,j} \frac{(-1)^j}{2j-t+1}\binom{2j-t+1}{k}B_{2j-t+1-k}, \quad 0\leq t\leq m,
\end{equation*}
where $B_1=\tfrac12$.
\end{myitemize}
\section{Polynomials \texorpdfstring{$\mathbf{P}^{m}_{a,b}(n)$}{Pm[a,b](n)} and their properties}
We'd like to begin our discussion from defined above polynomial $\mathbf{P}^{m}_{a,b}(n)$. Polynomial $\mathbf{P}^{m}_{a,b}(n)$ is $2m+1$ degree polynomial in $a,b,n$. Polynomial $\mathbf{P}^{m}_{a,b}(n)$ is defined as finite sum of $2m$ degree polynomial $\mathbf{L}_m(n,k)$ over $k$. Being a summation, by means of associativity the polynomials $\mathbf{P}^{m}_{a,b}(n)$ can be split
\begin{equation*}
\mathbf{P}^{m}_{a,b}(n)=\mathbf{P}^{m}_{b}(n)-\mathbf{P}^{m}_{a}(n).
\end{equation*}
Polynomials $\mathbf{P}^{m}_{a,b}(n)$ implicitly involve the polynomials $\mathbf{Q}^{r}_{a,b}(n), \ \mathbf{X}^{m}_{t}(a,b), \ \mathbf{H}_{m,t}(k)$ and common power sum $S_p(n)$, see Notation and conventions. In extended form the polynomials $\mathbf{P}^{m}_{a,b}(n)$ are
\begin{equation}
\label{p_all_forms}
\begin{split}
\mathbf{P}^{m}_{a,b}(n)
&=\sum_{a \leq k < b}\mathbf{L}_m(n,k)
 =\sum\limits_{k}\mathbf{A}_{m,k}\mathbf{Q}^{k}_{a,b}(n)
 =\sum\limits_{k}\mathbf{A}_{m,k}(\mathbf{Q}^{k}_{b}(n)-\mathbf{Q}^{k}_{a}(n)) \\
&=\sum_{k}\mathbf{X}^{m}_{k}(a,b) (-1)^{m-k} n^k
 =\sum_{k} (\mathbf{X}^{m}_{k}(b)-\mathbf{X}^{m}_{k}(a)) (-1)^{m-k} n^k \\
&=\sum_{k} \sum\limits_{j\geq k}(-1)^{2m+j-k} \mathbf{A}_{m,j} \binom{j}{k}(S_{2j-k}(b)-S_{2j-k}(a)) n^k \\
&=\sum_{k} (-1)^{2m-k} \sum_{\ell=1}^{2m-k+1} \mathbf{H}_{m,k}(\ell)(b^\ell - a^\ell)n^k. \\
\end{split}
\end{equation}
The last line of the expression \eqref{p_all_forms} clearly states why $\mathbf{P}^{m}_{a,b}(n)$ are polynomials in $a,b,n$. Let's show a few examples of polynomials $\mathbf{P}^{m}_{a,b}(n)$
\begin{equation*}
\begin{split}
\mathbf{P}^{0}_{a,b}(n)
&=-a+b.\\
\mathbf{P}^{1}_{a,b}(n)
& =-3 a^2 + 2 a^3 \\
&\quad + 3 b^2 - 2 b^3 \\
&\quad + 3 a n - 3 a^2 n \\
&\quad - 3 b n + 3 b^2 n. \\
\mathbf{P}^{2}_{a,b}(n)
&=- 10 a^3 + 15 a^4 - 6 a^5 \\
&\quad + 10 b^3 - 15 b^4 + 6 b^5 \\
&\quad + 15 a^2 n - 30 a^3 n + 15 a^4 n \\
&\quad - 15 b^2 n + 30 b^3 n - 15 b^4 n \\
&\quad - 5 a n^2 + 15 a^2 n^2 - 10 a^3 n^2 \\
&\quad + 5 b n^2 - 15 b^2 n^2 + 10 b^3 n^2. \\
\end{split}
\end{equation*}
\begin{equation*}
\begin{split}
\mathbf{P}^{2}_{a,b}(n)
&= \quad 7 a^2 - 28 a^3 + 70 a^5 - 70 a^6 + 20 a^7 \\
&\quad - 7 b^2 + 28 b^3 - 70 b^5 + 70 b^6 - 20 b^7 \\
&\quad - 7 a n + 42 a^2 n - 175 a^4 n + 210 a^5 n -70 a^6 n \\
&\quad + 7 b n - 42 b^2 n + 175 b^4 n - 210 b^5 n + 70 b^6 n \\
&\quad - 14 a n^2 + 140 a^3 n^2 - 210 a^4 n^2 + 84 a^5 n^2 \\
&\quad + 14 b n^2 - 140 b^3 n^2 + 210 b^4 n^2 - 84 b^5 n^2 \\
&\quad - 35 a^2 n^3 + 70 a^3 n^3 - 35 a^4 n^3 \\
&\quad + 35 b^2 n^3 - 70 b^3 n^3 + 35 b^4 n^3.
\end{split}
\end{equation*}
We consider the polynomials $\mathbf{P}^{m}_{a,b}(n)$ because of their usefulness in revealing the main topic of the work. By means of partial cases of the polynomial $\mathbf{P}^{m}_{a,b}(n)$ we establish a relation between the power sum $\mathbf{Q}^{r}_{a,b}(n)$ and Binomial theorem. For instance, odd powers of $n$ are
\begin{equation*}
n^{2m+1} = \mathbf{P}^{m}_{n}(n) = \sum\limits_{k}\mathbf{A}_{m,k}\mathbf{Q}^{k}_{n}(n), \quad m\geq 0.
\end{equation*}
Moreover, the Binomial expansion $(a+b)^{2m+1}$ of odd powers can be reached similarly
\begin{equation*}
(a+b)^{2m+1}=\sum_{k} \binom{2m+1}{k} a^{2m+1-k} b^k \equiv \mathbf{P}^{m}_{a+b}(a+b) \equiv \sum\limits_{k}\mathbf{A}_{m,k}\mathbf{Q}^{k}_{a+b}(a+b), \quad m\geq 0.
\end{equation*}
It clearly follows that Multinomial expansion of odd-powered $t$-fold sum $(a_1+a_2+\cdots+a_t)^{2m+1}$ can be reached by $\mathbf{P}^{m}_{a,b}(n)$ as well
\begin{equation*}
\begin{split}
(a_1+a_2+\cdots+a_t)^{2m+1}
&=\sum_{k_1+k_2+\cdots+k_t=2m+1}\binom{2m+1}{k_1, k_2,\ldots, k_t} \prod_{s=1}^{t}a_t^{k_t} \\
\\
&\equiv\mathbf{P}^{m}_{a_1+a_2+\cdots+a_t}(a_1+a_2+\cdots+a_t)\\
\\
&\equiv\sum_{k}\mathbf{A}_{m,k}\mathbf{Q}^{k}_{a_1+a_2+\cdots+a_t}(a_1+a_2+\cdots+a_t), \quad m\geq 0.
\end{split}
\end{equation*}

Since the $n^s = n^{[s \ \mathrm{is} \ \mathrm{even}]} n^{\lfloor (s-1)/2 \rfloor}$, it is easy to generalise previously obtained odd power identity for all exponents $s\in\mathbb{N}$
\begin{equation}
\label{all_power_identity}
n^s
= n^{[s \ \mathrm{is} \ \mathrm{even}]} \mathbf{P}^{\lfloor \tfrac{s-1}{2} \rfloor}_{n}(n)
= n^{[s \ \mathrm{is} \ \mathrm{even}]}\sum_{k}^{ \ }\mathbf{A}_{\lfloor \tfrac{s-1}{2} \rfloor, k}\mathbf{Q}^{k}_{n}(n), \quad s>0.
\end{equation}
The binomial expansion of $(a+b)^s$ for every integer $s>0$ is
\begin{equation*}
\begin{split}
(a+b)^s=\sum_{k} \binom{s}{k} a^{s-k} b^k
&\equiv(a+b)^{[s \ \mathrm{is} \ \mathrm{even}]} \mathbf{P}^{\lfloor \tfrac{s-1}{2} \rfloor}_{a+b}(a+b) \\
&\equiv(a+b)^{[s \ \mathrm{is} \ \mathrm{even}]}\sum_{k}^{ \ }\mathbf{A}_{\lfloor \tfrac{s-1}{2} \rfloor, k}\mathbf{Q}^{k}_{a+b}(a+b).
\end{split}
\end{equation*}
Now we are able to generalise the expression \eqref{all_power_identity} even more. For the $t$-fold $s$-powered sum $(a_1+a_2+\cdots+a_t)^s, \quad s>0$ we have following Multinomial expansion
\begin{equation*}
\begin{split}
(a_1+a_2+\cdots+a_t)^s
&=\sum_{k_1+k_2+\cdots+k_t=s}\binom{s}{k_1, k_2,\ldots, k_t} \prod_{\ell=1}^{t}a_\ell^{k_\ell}\\
\\
&\equiv (a_1+a_2+\cdots+a_t)^{[s \ \mathrm{is} \ \mathrm{even}]}
\mathbf{P}^{\lfloor \tfrac{s-1}{2} \rfloor}_{a_1+a_2+\cdots+a_t}(a_1+a_2+\cdots+a_t) \\
\\
&\equiv (a_1+a_2+\cdots+a_t)^{[s \ \mathrm{is} \ \mathrm{even}]}
\sum_{k}^{ \ }\mathbf{A}_{\lfloor \tfrac{s-1}{2} \rfloor, k}\mathbf{Q}^{k}_{a_1+\cdots+a_t}(a_1+\cdots+a_t).
\end{split}
\end{equation*}
\section{Relation between the polynomials \texorpdfstring{$\mathbf{P}^{m}_{a,b}(n)$}{Pm[a,b](n)} and convolution of power function \texorpdfstring{$f^r(n)$}{f_r(n)}}
Previously we have established a relation between the polynomials $\mathbf{P}^{m}_{a,b}(n)$ and Binomial theorem. In this section a relation between $\mathbf{P}^{m}_{a,b}(n)$ and convolution of the piecewise defined power function $f_{t}^{r}$ is established. To show that P implicitly involves the discrete convolution of piecewise defined power function $f_{t}^{r}$ let's refresh what P are
\begin{equation*}
P=\sum \mathbf{A Q}.
\end{equation*}
Meanwhile, the term Q is the power sum of the form
\begin{equation*}
\mathbf{Q} = \sum k^r (n-k)^r
\end{equation*}
It could be noticed immediately that Q differs from the discrete convolution of $f_{t}^{r}$ only in sense of boundary conditions of the summation. For instance, the discrete convolution of the piecewise defined power function $f_{t}^{r}$ is
\begin{equation*}
\begin{split}
(f_{t}^{r} \ast f_{t}^{r})[n]
&= \sum_{k}f_{r,t}(k)f_{r,t}(n-k) = \sum_{k}k^r(n-k)^r[k\geq t][n-k\geq t] \\
&= \sum_{k}k^r(n-k)^r[t\leq k \leq n-t].
\end{split}
\end{equation*}
It is now clear that discrete convolution $(f_{t}^{k} \ast f_{t}^{k})[n]$ of piecewise defined power function $f_{t}^{r}$ is a partial case of the power sum Q with $a=$ and $b=$, ie
\begin{equation*}
(f_{t}^{r} \ast f_{t}^{r})[n]  = \mathbf{Q}_{t,n-t+1}^r(n), \quad n\geq 1.
\end{equation*}
Therefore, the polynomials $\mathbf{P}^{m}_{a,b}(n)$ are in relation with discrete convolution of piecewise defined power function $f_{t}^{r}$ as follows
\begin{equation*}
\mathbf{P}^{m}_{t,n-t+1}(n)
=\sum\limits_{r}\mathbf{A}_{m,r} \mathbf{Q}_{t,n-t+1}^r(n)
\equiv \sum\limits_{r}\mathbf{A}_{m,r} (f_{t}^{r} \ast f_{t}^{r})[n], \quad n\geq 1.
\end{equation*}
Following this logic, we are able to find a relation between $P$ and discrete convolution of power function \texorpdfstring{$f_{t}^{r}$}{fr(n)}.
\subsection{Relation between Binomial theorem and convolution of power function \texorpdfstring{$f^r(n)$}{f_r(n)}}
As it is stated previously in (exp link), the polynomials P are able to be expressed in terms of convolution $(f_{r,t} \ast f_{r,t})[n]$ of $f_{r,t}(n)$. Consequently, by the equivalence between BT and P which is (4), the Binomial expansion could be expressed in terms of convolution $n_{\geq t}^r \ast n_{\geq t}^r$ as well,
\begin{equation*}
\begin{split}
(a+b)^{2m+1}=\sum_{r} \binom{2m+1}{r} a^{2m+1-r} b^r
&\equiv -1+\mathbf{P}^{m}_{a+b+1}(a+b) \\
&= -1+\sum_{r}\mathbf{A}_{m,r}\mathbf{Q}^{r}_{a+b+1}(a+b)\\
&= -1+\sum\limits_{r}\mathbf{A}_{m,r} (f^{r} \ast f^{r})[a+b].
\end{split}
\end{equation*}
\section{Derivation of the coefficients \texorpdfstring{$\mathbf{A}_{m,r}$}{A(m,r)}}
Assuming that
\begin{equation*}
n^{2m+1} = \mathbf{P}^{m}_{n}(n) = \sum\limits_{k}\mathbf{A}_{m,k}\mathbf{Q}^{k}_{n}(n), \quad m\geq 0.
\end{equation*}
The coefficients $\mathbf{A}_{m,r}$ could be evaluated expanding $\mathbf{Q}^{k}_{n}(n)=\sum_{k=0}^{n-1}k^r(n-k)^r$ and using Faulhaber's formula $\sum _{k=1}^{n}k^{p}=\tfrac{1}{p+1}\sum _{j=0}^{p}{p+1 \choose j}B_{j}n^{p+1-j}$, we get
\begin{equation}\label{proof1}
\begin{split}
&\sum_{k=0}^{n-1}k^r(n-k)^r\\
&=\sum_{k=0}^{n-1} k^r \sum_{j} (-1)^j\binom{r}{j} n^{r-j}k^{j}=\sum_{j} (-1)^j\binom{r}{j} n^{r-j}\left(\sum_{k=0}^{n-1}k^{r+j}\right)\\
&=\sum_{j} \binom{r}{j} n^{r-j}\frac{(-1)^j}{r+j+1}\left[\sum_{s}\binom{r+j+1}{s}B_{s}n^{r+j+1-s}-B_{r+j+1}\right]\\
&=\sum_{j,s}\binom{r}{j}\frac{(-1)^j}{r+j+1}\binom{r+j+1}{s}B_{s}n^{2r+1-s}-\sum_{j} \binom{r}{j}\frac{(-1)^j}{r+j+1}B_{r+j+1}n^{r-j}\\
&=\sum_{s}\underbrace{\sum_{j}\binom{r}{j}\frac{(-1)^j}{r+j+1}\binom{r+j+1}{s}}_{S(r)}B_{s}n^{2r+1-s}-\sum_{j} \binom{r}{j}\frac{(-1)^j}{r+j+1}B_{r+j+1}n^{r-j}
\end{split}
\end{equation}
where $B_s$ are Bernoulli numbers and $B_1=\tfrac12$.
Now, we notice that
\begin{equation*}
S(r)=\sum_{j} \binom{r}{j}\frac{(-1)^j}{r+j+1}\binom{r+j+1}{s}
=\begin{cases}
\frac{1}{(2r+1)\binom{2r}r}, & \text{if } s=0;\\
\frac{(-1)^r}{s}\binom{r}{2r-s+1}, & \text{if } s>0.
\end{cases}
\end{equation*}
In particular, the last sum is zero for $0<s\leq r$. Therefore, expression (\hyperref[proof1]{3.1}) takes the form
\begin{equation*}
\begin{split}
\sum_{k=0}^{n-1}k^r(n-k)^r
&=\frac{1}{(2r+1)\binom{2r}r}n^{2r+1}+\underbrace{\sum_{s\geq 1}\frac{(-1)^r}{s}\binom{r}{2r-s+1}B_{s}n^{2r+1-s}}_{(\star)}\\
&-\underbrace{\sum_{j} \binom{r}{j}\frac{(-1)^j}{r+j+1}B_{r+j+1}n^{r-j}}_{(\diamond)}
\end{split}
\end{equation*}
Hence, introducing $\ell=2r+1-s$ to $(\star)$ and $\ell=r-j$ to $(\diamond)$, we get
\begin{equation*}
\begin{split}
\sum_{k=0}^{n-1}k^r(n-k)^r
&=\frac{1}{(2r+1)\binom{2r}r}n^{2r+1}+\sum_{\ell}\frac{(-1)^r}{2r+1-\ell}\binom{r}{\ell}B_{2r+1-\ell}n^{\ell}\\
&-\sum_{\ell} \binom{r}{\ell}\frac{(-1)^{j-\ell}}{2r+1-\ell}B_{2r+1-\ell}n^{\ell}\\
&=\frac{1}{(2r+1)\binom{2r}r}n^{2r+1}+2\sum_{\mathrm{odd} \ \ell}\frac{(-1)^r}{2r+1-\ell}\binom{r}{\ell}B_{2r+1-\ell}n^{\ell}
\end{split}
\end{equation*}
Using the definition of $\mathbf{A}_{m,r}$ coefficients, we obtain the following identity for polynomials in $n$
\begin{equation}\label{proof2}
\sum_{r}\mathbf{A}_{m,r}\frac{1}{(2r+1)\binom{2r}r}n^{2r+1}
+2\sum_{r, \ \mathrm{odd} \ \ell}\mathbf{A}_{m,r}\frac{(-1)^r}{2r+1-\ell}\binom{r}{\ell}B_{2r+1-\ell}n^{\ell}\equiv n^{2m+1}
\end{equation}
Taking the coefficient of $n^{2m+1}$ in (\hyperref[proof2]{3.2}) we get $\mathbf{A}_{m,m}=(2m+1)\binom{2m}m$ and taking the coefficient of $n^{2d+1}$ for an integer $d$ in the range $m/2 \leq d < m$, we get $\mathbf{A}_{m,d}=0$. Taking the coefficient of $n^{2d+1}$ for $d$ in the range $m/4 \leq d < m/2$, we get
\begin{equation*}
\mathbf{A}_{m,d} \frac{1}{(2d+1)\binom{2d}{d}} + 2 (2m+1) \binom{2m}{m} \binom{m}{2d+1} \frac{(-1)^m}{2m-2d} B_{2m-2d} = 0,
\end{equation*}
i.e,
\begin{equation*}
\mathbf{A}_{m,d} = (-1)^{m-1} \frac{(2m+1)!}{d!d!m!(m-2d-1)!}\frac{1}{m-d} B_{2m-2d}.
\end{equation*}
Continue similarly, we can express $\mathbf{A}_{m,d}$ for each integer $d$ in range $m/2^{s+1}\leq d< m/2^s$ (iterating consecutively $s=1,2...$) via previously determined values of $\mathbf{A}_{m,j}$ as follows
\begin{equation*}
\mathbf{A}_{m,d} = (2d+1)\binom{2d}{d} \sum_{j\geq 2d+1} \mathbf{A}_{m,j} \binom{j}{2d+1} \frac{(-1)^{j-1}}{j-d} B_{2j-2d}.
\end{equation*}
Thus, for every $(n,m)\in\mathbb{N}$ holds
\begin{equation*}
n^{2m+1}=\sum_{r=0}^{m}\mathbf{A}_{m,r}\sum_{k=0}^{n-1}k^r(n-k)^r.
\end{equation*}
\section{Acknowledgements}
We would like to thank to Dr. Max Alekseyev (Department of Mathematics and Computational Biology,
George Washington University) for sufficient help in the derivation of $\mathbf{A}_{m,r}$ coefficients,
Also, we'd like to thank to OEIS editors Michel Marcus, Peter Luschny, Jon E. Schoenfield and others
for their patient, faithful volunteer work and for their useful comments and suggestions during
the editing of the sequences connected with this manuscript.
\section{Conclusion}
\bibliographystyle{alpha}
\bibliography{references}
\end{document}
